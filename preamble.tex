% Every document has "big" preambles. That is a little bit 
% messy (my oppinion), so this file pretends to be the 
% \usepackage and the config of the package "file". So 
% I keep everything organized.


% Set lang (support for xelatex and lualatex)
\usepackage{fontspec}
\usepackage{polyglossia}
\setdefaultlanguage{spanish}

% Enhanced font: Computer Modern Unicode
\setmainfont{CMU Serif}
\setsansfont{CMU Sans Serif}
\setmonofont{CMU Typewriter Text}

% Bibliography stuff
\usepackage[backend=biber, style=authoryear-icomp]{biblatex}
\addbibresource{main.bib}

% Math stuff
\usepackage{amsmath}

% Colors
\usepackage{xcolor}
\definecolor{soft_blue}{rgb}{0.39, 0.58, 0.93} % dark_blue
\definecolor{soft_red}{rgb}{0.8, 0.31, 0.36} % red
\definecolor{soft_green}{RGB}{100,180,100} % green
\definecolor{url}{rgb}{0.45, 0.29, 0.58} % purple

% Urls and hyperref
\usepackage[colorlinks=true,
    linkcolor=url,
    urlcolor=soft_blue,
    citecolor=soft_blue,
    hypertexnames=false]{hyperref}

% SI units (V,A,m,kg)
\usepackage{siunitx}
\sisetup{
  locale = FR,         % El estilo francés usa la coma decimal, igual que el español técnico
  separate-uncertainty = true, % Útil para mediciones: 10,0 +- 0,1
  per-mode = symbol    % Para usar V/m en lugar de V m^{-1}
}
\DeclareSIUnit{\div}{div}

% For better tables
\usepackage{booktabs}

% captions and references
\usepackage{caption}

% pgfplots and tikz for graphs
\usepackage[mode=buildnew]{standalone}
\usepackage{tikz}
\usepackage{circuitikz}
\usepackage{pgfplots}
\usetikzlibrary{calc, positioning}
\usetikzlibrary{3d, arrows.meta, decorations.markings, backgrounds, fadings}
% this line is recomended by the pgfplots lastes stable version
% see https://pgfplots.sourceforge.net/pgfplots.pdf
\pgfplotsset{compat=1.18}

\usepackage{tcolorbox}
