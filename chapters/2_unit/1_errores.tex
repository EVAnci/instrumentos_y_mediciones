\section{Tipos de error}

Considere una magnitud cuyo valor real es \(x\) y su valor medido es \(X\).

\begin{definition}[Error absoluto]
  Se define como la diferencia entre el valor exacto y el aproximado:
  \[
    \varepsilon_x = x - X
  \]
\end{definition}

\begin{example}
  Supongamos que hemos medido el diámetro de una esfera con una regla y obtenemos \(X=\qty{1}{\centi\meter}\). Si el valor real fuese \(x=\qty{1.03}{\centi\meter}\), el error absoluto sería:
  \[
    \varepsilon_x = \qty{1.03} - \qty{1} = +\qty{0.03}{\centi\meter}
  \]
  \label{ej_error_absoluto}
\end{example}

Como es inusual conocer el valor real, se utiliza la Cota de error absoluto (\(\Delta_x^\circ\)), que define un intervalo donde se asegura que está el valor real:
\begin{equation}
  x = X \pm \Delta_x^\circ
\end{equation}

\begin{definition}[Error relativo]
  Indica la calidad de la medida. Es el cociente entre el error absoluto y el valor exacto:
  \[
    \rho_x = \frac{\varepsilon_x}{x} \approx \frac{\varepsilon_x}{X}
  \]
  Frecuentemente se expresa como porcentaje: \(\rho\% = \rho_x \cdot 100\).
\end{definition}

\section{Clasificación general de los errores}

Los errores se clasifican según su origen y comportamiento en tres grandes grupos: groseros, sistemáticos y accidentales.

\subsection{Error Grosero}
Son equivocaciones graves, generalmente humanas, que producen valores atípicos o ``disparates'' en las mediciones.
\begin{itemize}
  \item Mala lectura de escala (ej. leer 100 en lugar de 10).
  \item Selección incorrecta del instrumento o rango.
  \item Errores de conexión.
\end{itemize}
Solución: Estos errores no se corrigen matemáticamente; la medición debe descartarse y repetirse.

\subsection{Error Sistemático}
Se caracterizan por repetirse constantemente en magnitud y signo bajo las mismas condiciones. Son predecibles y, por tanto, calculables y corregibles.

\subsubsection*{1. Error sistemático del método (o de inserción)}
Ocurre porque los instrumentos no son ideales e interactúan con el circuito, modificando la variable que intentan medir.

Analicemos la medición de una resistencia incógnita \(R\) mediante un Voltímetro (con resistencia interna \(R_v\)) y un Amperímetro (con resistencia interna \(R_a\)). Existen dos configuraciones posibles:

\begin{figure}[!ht]
  \centering
  \begin{subfigure}[b]{0.48\textwidth}
    \centering
    \begin{circuitikz}[american]
      \def\ll{2.5} 
      \draw (0,0) to[battery2,l=E,invert] (0,3) to[rmeterwa,t=A,l=\(I_m\)] (\ll,3) to[rmeterwa,t=V,l=\(U_m\)] (\ll,0) -- (0,0);
      \draw (\ll,3) -- (4,3) to[R=\(R\)] (4,0) -- (\ll,0);
    \end{circuitikz}
    \caption{Conexión corta.}
    \label{fig_conexion_corta}
  \end{subfigure}
  \hfill
  \begin{subfigure}[b]{0.48\textwidth}
    \centering
    \begin{circuitikz}[american]
      \def\ll{1.5} 
      \draw (0,0) to[battery2,l=E,invert] (0,3) -- (\ll,3) to[rmeterwa,t=A,l=\(I_m\)] (4,3) to[R=\(R\)] (4,0) -- (0,0);
      \draw (\ll,3) to[rmeterwa,t=V,l=\(U_m\)] (\ll,0);
    \end{circuitikz}
    \caption{Conexión larga.}
    \label{fig_conexion_larga}
  \end{subfigure}
  \caption{Medición de una resistencia.}
\end{figure}

\textbf{A) Conexión Corta (Fig. \ref{fig_conexion_corta}):}
El voltímetro mide bien la tensión sobre \(R\), pero el amperímetro mide la corriente de la resistencia \emph{más} la corriente que se fuga por el voltímetro.
\[
  R_{\text{medida}} = \frac{U_m}{I_m} \approx \frac{R}{1 + R/R_v}
\]
Como el voltímetro real no tiene resistencia infinita (\(R_v < \infty\)), la resistencia medida será menor a la real. Se recomienda usar esta conexión para medir resistencias pequeñas (\(R \ll R_v\)).

\textbf{B) Conexión Larga (Fig. \ref{fig_conexion_larga}):}
El amperímetro mide bien la corriente de \(R\), pero el voltímetro mide la caída de tensión en \(R\) \emph{más} la caída en el amperímetro.
\[
  R_{\text{medida}} = R + R_a
\]
Como el amperímetro real tiene resistencia (\(R_a > 0\)), el valor medido será mayor al real. Se recomienda para medir resistencias grandes (\(R \gg R_a\)).

\subsubsection*{2. Error instrumental y ambiental}
Son debidos a imperfecciones constructivas (rozamientos, resortes envejecidos) o condiciones externas (temperatura, campos magnéticos). El fabricante suele garantizar un error máximo mediante la Clase del instrumento (ver Cap. \ref{chpt:introduccion}). Si se opera fuera de las condiciones nominales (ej. temperatura excesiva), el error aumentará sistemáticamente.

\subsection{Error Accidental (o Aleatorio)}
Son pequeñas variaciones impredecibles que ocurren al azar en ambos sentidos (positivo y negativo). A diferencia de los sistemáticos, no se pueden eliminar, pero se pueden minimizar mediante tratamiento estadístico (promedio de múltiples mediciones).

En instrumentos analógicos, los errores accidentales más comunes en la lectura son:

\begin{description}
  \item[Error de Paralaje:] Ocurre cuando la línea de visión del observador no es perpendicular a la escala, proyectando la aguja en una posición falsa. Se mitiga con el espejo antiparalaje (la imagen reflejada de la aguja debe quedar oculta tras la aguja real) o usando índices luminosos (Fig. \ref{fig_indice_luminoso}).
  
  \item[Límite de resolución (Poder separador):] El ojo humano tiene un límite físico. No puede distinguir dos puntos separados por menos de \(\approx \qty{0.1}{\milli\meter}\) a distancia de lectura normal. Esto significa que, aunque la aguja se mueva infinitesimalmente, el ojo la verá quieta. Esto introduce una incertidumbre de lectura inevitable, estimada típicamente en \(1/5\) o \(1/10\) de la división más pequeña de la escala.
\end{description}

\begin{figure}[!ht]
  \centering
  \includegraphics[height=5cm]{chapters/2_unit/media/bin/indice_luminoso.png}
  \caption{Galvanómetro de índice luminoso, exento de error de paralaje.}
  \label{fig_indice_luminoso}
\end{figure}
