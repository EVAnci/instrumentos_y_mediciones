\section{Glosario y vocabulario}

A continuación se enumera el vocabulario y las definiciones básicas que se requieren a lo largo del libro.

\begin{enumerate}
  \item Magnitud (\(X\)): se mide en la unidad física (\unit{\volt}, \unit{\ampere}, \unit{\ohm}). Usualmente en literatura ingenieril, se utiliza la letra \(U\) para hablar de diferencia de potencial (o tensión) como variable, \(I\) la corriente y \(R\) la resistencia.\footnote{Note que las unidades son rectas, mientras que las variables itálicas.}

  \item Exactitud: La proximidad del valor medido con el real.

  \item Precisión: La repetibilidad de las lecturas en el mismo. No siempre un instrumento preciso significa que sea exacto. A la inversa un instrumento exacto ha de ser siempre preciso.

  \item Sensibilidad de un instrumento: Relación efecto/causa. Esto se refiere puramente a la transferencia del dispositivo. Causa es la magnitud física que quiere medir (\(X\)). Por ejemplo: Tensión, Corriente o Temperatura. Efecto es la respuesta observable del instrumento (\(Y\)). Por ejemplo desplazamiento de la aguja en \unit{\milli\metre}.
    \[
      S_i = \frac{\Delta\text{Efecto}}{\Delta\text{Causa}} = \frac{\Delta Y}{\Delta X}
    \]
    Por ejemplo: tiene un termómetro de mercurio y por cada \qty{1}{\degreeCelsius} que sube la temperatura (causa), la columna de mercurio sube \qty{2}{\centi\metre} (efecto), la sensibilidad del instrumento es de \qty{2}{\centi\metre\per\degreeCelsius}.

    Si un instrumento (por ejemplo de imán permanente y bobina móvil) tiene una ley de respuesta
    \[
      I=k\alpha
    \]
    se puede ver claramente que para una pequeña variación de corriente \(\delta I\) habrá una pequeña variación de deflexión \(\delta\alpha\). Y a cada pequeño incremento de corriente le corresponderá el mismo cambio en la deflexión. Entonces se dice que la sensibilidad del instrumento es constante:
    \begin{equation}\label{eq_sensibilidad_constante}
      S=\frac{\delta \alpha}{\delta I}
    \end{equation}

    Por otro lado, si la deflexión responde a una ley alineal, entonces el cociente \eqref{eq_sensibilidad_constante} no es constante y varía punto a punto, donde su expresión matemática será 
    \[
      \lim_{\Delta I \to 0} \frac{\Delta \alpha}{\Delta I} \approx \frac{d\alpha}{dI}
    \]

  \item Sensibilidad de una técnica de medida: Se define como el cociente entre la magnitud total y el cambio más pequeño que es capaz de detectar. Por ejemplo si desea medir con un calibre muy sensible, pero lo utiliza para medir una pieza vibratoria o en un ambiente con muy mala iluminación, su capacidad para distinguir una pequeña variación disminuye. La \emph{técnica} se vuelve menos sensible aunque el instrumento siga siendo el mismo.
    
  \item Medir: Significa comparar una magnitud correspondiente con una unidad apropiada. El valor de la medida queda expresado como el producto entre el valor medido y la unidad correspondiente.

  \item Deflexión (\(\alpha\)): Cantidad de divisiones o grados que se desvía la aguja indicadora sobre una escala de un determinado instrumento. La deflexión suele ser un ángulo de giro o la cantidad de divisiones\footnote{Para la unidad no estándar ``divisiones'' que recorre la aguja de un instrumento cuando se deflexiona usaremos la unidad \unit{\div}.} que ha recorrido la aguja. Se denomina \(\alpha\) al valor actual en donde se detiene la aguja y \(\alpha_\text{max}\) al límite físico del instrumento (fondo de la escala).

    Un ejemplo: si tiene un voltímetro con una escala de \qtyrange{0}{50}{\div} y lo usa para medir en un rango de \qty{0}{\volt} a \qty{500}{\volt}, entonces \(\alpha_{\text{max}}=\qty{50}{\div}\).\footnote{Vea, que se dice \qtyrange{0}{50}{\div} ya que hay instrumentos que permiten deflexión hacia dos lados, de modo que, a modo de ejemplo, un voltímetro puede indicar \qtyrange{-25}{25}{\volt} y tener \qtyrange{-10}{10}{\div}.} Si al realizar una medida la aguja se clava a la mitad, la \emph{deflexión} es de \qty{25}{\div}, consecuentemente la lectura será de \qty{250}{\volt}.

  \item Campo nominal de referencia: Indica el \emph{rango} de un determinado parámetro en el cual el instrumento mantiene su clase (exactitud). Por ejemplo un voltímetro para corriente alterna que tiene una leyenda indicando \qtyrange{40}{60}{\hertz} quiere decir que su exactitud en la medida se mantiene siempre y cuando se respete ese rango de la frecuencia.

  \item Clase: se define como el error absoluto máximo \(E_{\text{max}}\) que puede cometer el instrumento en cualquier parte de su escala, referido a su alcance expresado en valor porcentual:
    \[
      c = \frac{E_{\text{max}}}{\text{Alcance}}\cdot 100
    \]
    La definición de alcance se presentará en las próximas definiciones.

  \item Cuadrante: es la cara frontal del instrumento que permite interpretar la magnitud medida. Contiene la escala graduada que es el conjunto de trazos y números que representan los valores de magnitud, la aguja indicadora (o índice) que se desplaza sobre el cuadrante para indicar el valor medido y un espejo antiparo que es una franja de espejo situada bajo la escala que sirve para evitar el error de paralaje.

    El cuadrante del instrumento puede contener un número acompañado con un símbolo indicando el principio de funcionamiento. Dicho número representa la clase del instrumento que, cuanto menor sea, mejor será la exactitud de las medidas. Si el instrumento no cuenta con el número de clase, el fabricante no garantiza la clase del instrumento.

  \item Rango de medida: Se define así al tramo de la escala en el cuadrante donde las medidas son confiables. 
    \[
      \text{Rango de medida} = [X_{\text{min}},X_{\text{max}}]
    \]
    El valor máximo del rango de medida (máxima magnitud: \(X_{\text{max}}\)) queda definido como el \emph{alcance} del instrumento. 

    Por ejemplo si se tiene un voltímetro que presenta en la escala del cuadrante \qty{0}{\volt} a \qty{500}{\volt} ese es su rango de medida (todos los valores entre \qtyrange{0}{500}{\volt}).

    Si el instrumento responde a una ley de deflexión lineal, entonces el rango de medida será coincidente con el alcance. Para instrumentos cuya ley de deflexión es cuadrática, la escala será lineal. De modo que el fabricante tratará (mediante dispositivos constructivos) de que sea así. No obstante, estos últimos aparatos suelen representar los primeros valores de la escala de forma comprimida, debido a la imposibilidad de su correcta calibración. Así, el alcance del instrumento no considera correctas (o con la exactitud dada por la clase) a los primeros valores más comprimidos. 

  \item Rango de deflexión: similar al rango de medida, pero relacionado a la deflexión. Se define como 
    \[\text{Rango de deflexión}=[\alpha_\text{min},\alpha_\text{max}]\]
    Así, corresponde a el movimiento que puede tener la aguja del instrumento sobre el cuadrante.

    No necesariamente el rango de deflexión y el rango de medida coinciden en su totalidad. Algunos instrumentos presentan un pequeño desborde al llegar a los límites de deflexión, garantizando ser precisos en el rango de medida.

  \item Margen de la indicación: Toda la escala del cuadrante.

  \item Constante de lectura: Se define como el cociente entre el alcance (\(X_{\text{max}}\)) y la máxima deflexión en divisiones (\(\alpha_{\text{max}}\)).
    \[
      C_E = \frac{\text{Alcance}}{\alpha_{max}}
    \]
    Cuando la aguja se deflexiona una cantidad \(\alpha\), la magnitud que está midiendo será:
    \[
      X_{\text{medido}} = \alpha \, C_E
    \]

  \item Consumo propio: Es la potencia absorbida por el propio instrumento para provocar su propia deflexión. Un voltímetro tendrá idealmente una resistencia infinita. Un amperímetro, idealmente, tendrá una resistencia nula. 

    Un voltímetro es conectado usualmente en paralelo\footnote{Recuerde que \(U\) representa la tensión.}. Así, su potencia de consumo será menor siempre que la resistencia sea mayor: \(P=\frac{U^2}{R}\). Para un amperímetro, que se conecta en serie, su potencia de consumo será menor siempre y cuando la resistencia sea menor: \(P=I^2R\).

  \item Resolución instrumental: es la variación de magnitud que provoca un mínimo cambio apreciable en la medición. Este valor suele valer desde \(1/5\) a \(1/10\) de división. Por ejemplo en un voltímetro con un rango de medida de \qtyrange{0}{500}{\volt} y \qty{50}{\div}, asumiendo que el voltímetro tiene una resolución instrumental de \(1/5\) entonces el mínimo cambio apreciable en la medición es de \qty{2}{\volt}.

  \item Sobrecarga: Es la máxima cantidad destructiva que tolera el instrumento sobre la máxima cantidad nominal (alcance).
    \[\text{Sobrecarga}=\frac{X_{\text{destructiva}}}{X_{\text{max}}}~ 100\%\]
    Si un voltímetro tiene un alcance de \qty{100}{\volt} y una sobrecarga del \(150\%\) significa que el voltímetro puede tolerar \qty{150}{\volt} sin destruirse.

\end{enumerate}

Antes de continuar con las unidades de medida, se realiza una distinción entre fuerza electromotriz y potencial eléctrico, ya que, suele ser motivo de confusión.

\subsection{Fuerza Electromotriz (FEM) y Potencial Eléctrico}

En el estudio de la metrología eléctrica, es crucial distinguir entre la energía suministrada al sistema y la energía disipada por los componentes del circuito.

\begin{definition}(FEM)
La Fuerza Electromotriz (FEM) se define como el trabajo total realizado por una fuente de energía (química, magnética o térmica) por unidad de carga positiva que atraviesa dicha fuente de un terminal de menor potencial a uno de mayor potencial. A pesar de su nombre histórico, no es una fuerza mecánica, sino una magnitud escalar de energía:
\begin{equation}
\mathcal{E} = \frac{dW}{dq}
\end{equation}
Donde:
\begin{itemize}
  \item \(\mathcal{E}\) es la FEM medida en Voltios (\(V\)).
  \item \(dW\) es el trabajo realizado sobre la carga en Julios (\(J\)).
  \item \(dq\) es la carga eléctrica en Coulombs (\(C\)).
\end{itemize}
\end{definition}

Es importante diferenciar el potencial eléctrico de la fuerza electromotriz. Aunque ambas magnitudes se miden en voltios, su origen físico y comportamiento en la instrumentación difieren significativamente. El cuadro \ref{tab_fem_vs_potencial} muestra una síntesis de las diferencias principales entre ambos conceptos.

\begin{table}[ht]
  \centering
  \begin{tabular}{p{0.45\textwidth}p{0.45\textwidth}}
    \toprule
    \textbf{Fuerza Electromotriz (\(\mathcal{E}\))} & \textbf{Diferencia de Potencial (\(V\))} \\ 
    \midrule
    Es la causa: energía que entra al circuito. & Es el efecto: energía que se consume o cae entre dos puntos. \\ 
    Se mantiene incluso si el circuito está abierto (corriente cero). & Es nula si no hay corriente o una fuente activa conectada. \\ 
    Está asociada a campos no conservativos dentro de la fuente. & Está asociada a campos electrostáticos conservativos. \\ 
    \bottomrule
    \end{tabular}
  \caption{Comparativa entre FEM y Diferencia de Potencial}
  \label{tab_fem_vs_potencial}
\end{table}

Desde la perspectiva de la medición, un voltímetro conectado a los bornes de una fuente no mide necesariamente su FEM, sino el voltaje terminal (\(V_t\)). Debido a la resistencia interna (\(R_i\)) de cualquier fuente real, la relación se expresa como:
\begin{equation}
V_t = \mathcal{E} - I \cdot R_i
\end{equation}
En condiciones de circuito abierto (\(I = 0\)), el instrumento de medición ideal registraría \(V_t = \mathcal{E}\). Por el contrario, en condiciones de carga, la caída de potencial interna \(I \cdot R_i\) reduce el voltaje disponible para el resto del sistema.

Con estas definiciones y vocabulario puede comenzar a interpretar conceptos fundamentales como las definiciones de las unidades de medida.
