\section{Instrumentos IPBM}

Son los llamados comúnmente de ``bobina móvil'' o ``tipo D'Arsonval'' y son los más utilizados en el campo de la corriente continua. Se basan en la acción motriz ejercida por el campo de un imán permanente sobre una bobina recorrida por una corriente continua.

Básicamente el instrumento está constituido por dos sistemas: uno fijo -el circuito magnético- y otro móvil -el cuadro de la bobina-.

El cuadro móvil puede estar sustentado por: pivotes, hilo o bien hilo tensado. Los resortes de bronce fosforoso sirven para llevar la corriente a la bobina móvil y crear la cupla directriz que es proporcional al ángulo de rotación del cuadro.

Los lados activos de la bobina -los paralelos al eje- se encuentran en el entrehierro existente, entre las expansiones polares de un imán permanente y un tambor cilíndrico fijo de hierro dulce. Las superficies enfrentadas de las expansiones polares y del núcleo son cilíndricas, de manera tal de obtener un campo de inducción uniforme y de dirección radial en toda la zona en el cual se puede mover la bobina.

\subsubsection*{Circuito magnético}

Siguiendo la figura \ref{fig:ipbm_esquema} circuito magnético del sistema D'Arsonval está constituido por:
\begin{enumerate}
  \item Iman permanente 
  \item Expansiones polares de hierro dulce
  \item Núcleo de hierro dulce
  \item Derivador magnético de hierro dulce % este no entendí muy bien que hace o para qué es
\end{enumerate}
\begin{figure}[!ht]
  \centering
  \includegraphics[height=5cm]{chapters/3_unit/media/bin/ipbm_esquema.png}
  \caption{Los números indicados corresponden a cada elemento enumerado.}
  \label{fig:ipbm_esquema}
\end{figure}

Téngase presente que todo el circuito magnético es fijo y su misión es la de crear en un entrehierro reducido un flujo radial e inducción constante a fin de obtener una distribución uniforme de escala.

El derivador magnético por el que circula parte del flujo disponible, restándolo del entre hierro en una fracción regulable a voluntad por adecuado desplazamiento, permite llevar la inducción $B$ a un valor perfectamente determinado a los fines de lograr en la fábrica que con la corriente del alcance, la aguja coincida con la última división de la escala.

También sirve para compensar el eventual debilitamiento del imán a través del tiempo: si el derivador se aleja, es menor el flujo que por él deriva, con lo que se refuerza el flujo del entrehierro, restituyéndolo al valor original. Con un circuito magnético que rodea la bobina móvil y el núcleo de hierro dulce, se logra un instrumento de amplia escala de medición, entre \qtyrange{250}{300}{\degree} en lugar de los \qtyrange{90}{120}{\degree} que se obtiene en los instrumentos comunes.

Para esta construcción especial solamente un lado de la bobina móvil es activo.

\subsubsection*{Ventajas de IPBM}

El instrumento de imán permanente y bobina móvil tiene las siguientes cualidades 
\begin{itemize}
  \item Elevada sensibilidad
  \item Consumo bajo
  \item Alto valor de la cifra de mérito (relación entre la cupla motora y el peso del rotor)
  \item Escala uniforme y con posibilidades de hacerla extendida (hasta \qty{300}{\degree})
  \item Poca influencia de los campos magnéticos externos
  \item Posibilidades de modificar las escalas, ampliar fácilmente el rango de medida.
\end{itemize}

\subsection{Ley de respuesta}

Para deducir la ley de respuesta del instrumento, consideremos una bobina de $N$ espiras. Al circular una corriente $I$ por el cuadro, aparece una fuerza actuante sobre los lados activos de la bobina:
\[
  F=BNIl
\]
siendo $l$ el lado activo de la bobina. La figura \ref{fig_par_ipbm} muestra una representación gráfica de la situación.
\begin{figure}[!ht]
  \centering
  \includegraphics[height=5cm]{chapters/3_unit/media/bin/ley_de_respuesta.png}
  \caption{}
  \label{fig_par_ipbm}
\end{figure}

Si $a$ es el ancho de la bobina, la cupla motriz será:
\[
  C_m = BNIl\alpha
\]
De esta expresión denominamos al producto $BNl\alpha$ constante motora $G$. Como ya se explicó en el estudio de las cuplas, una vez superado el transitorio el equilibrio final se logra cuando $C_m=C_d$. Entonces
\begin{gather*}
  GI=Kr\alpha \\ 
  \alpha= \frac{Gi}{K_r}
\end{gather*}
A la relación $G/K_r$ se la llama sensibilidad instrumental $S_i$:
\[
  \alpha = S_i I
\]
Esta nos dice que ante la presencia de un campo magnético radial y uniforme, la bobina móvil reacciona provocando una cupla y por ende una deflexión que es directamente proporcional a la corriente que circula por ella. Esta conclusión implica además que el instrumento posee una polaridad identificada en los bornes, (cuando opera con corriente continua). Si intencional o accidentalmente se invierte la polaridad, es evidente que la cupla cambiará de sentido, sin poder efectuar lectura alguna si el instrumento no posee escala con cero al centro.

\subsubsection*{Campo uniforme}

Si el campo magnético es constante, uniforme y paralelo el flujo concatenado por el cuadro móvil será función senoidal del ángulo de giro:
\[
  \varphi = \psi_\text{max} \sin(\alpha)
\]
La expresión general de la deflexión será 
\[
  \alpha = \frac{G}{K_r} I\cos(\alpha)
\]
como se desprende de la última expresión la ley sigue siendo lineal, es decir que responde proporcionalmente a la corriente. En cuanto a la ley de distribución de la escala dependerá ahora del coseno de la deflexión, obteniéndose una distribución de trazos casi lineal en la primera parte para comprimirse en la segunda.

% El documento presenta un gráfico, pero no logro descifrar que función ha graficado para hacerla en tikz con el entorno axis... Para darte una idea, es como una rama de una parábola (en particular la positiva).

Este tipo de instrumento es utilizado en los llamados de escala ampliada y tienen aplicación cuando se desean medir corrientes o tensiones cuyo valor medio se ubique aproximadamente en la mitad de la escala y con transitorios con picos elevados que por la particularidad constructiva del instrumento es posible detectar sin dañarlo.

\subsubsection*{Escala logarítmica}

Nuevamente la única manera de modificar la escala, apartándola de la linealidad, es hacer que uno de los factores de la constante motora $G$ sea variable. El único factor fácilmente variable es la inducción $B$ en el entrehierro. Así si se desea una escala logarítmica el valor de $B$ deberá ser proporcional a $\ln(i/i)$, lo que se consigue con una forma conveniente de los polos del imán, que hace que para bajos valores de $\alpha$ el valor de $B$ sea grande. % Aquí me parece que \ln(i/i) es un error ¿no?

Este tipo de instrumento con escala logarítmica es utilizado en el campo de las mediciones luminotécnicas y de sonido.


\subsubsection*{Amperímetros}

El instrumento de imán permanente y bobina móvil es muy sensible a la corriente. Esta corriente entra y sale del cuadro móvil por los resortes que cumplen la misión de cupla antagónica (espirales o cinta tensada). Con intensidades del orden de los \qtyrange{15}{20}{\milli\ampere} estos resortes alcanzan la temperatura máxima admisible. Por esta razón ese es el valor máximo de la intensidad que puede medirse con un aparato así construido y que se llama miliamperímetro. Si el alcance es del orden de los microamperes, el nombre que recibe es microamperímetro.
\begin{figure}[!ht]
  \centering
  \begin{circuitikz} 
    \draw (0,0) to[short, i=$I$, -*] (1,0) node[below] {A}
    (1,0) to[generic, l=$R_\text{shunt}$, i_=$I_s$] (4,0) to[short,-*] node[below] {B}
      (4,0) -- (5,0);
    \draw (1,0) -- (1,1.5)
      (1,1.5) to[rmeterwa, t=A, l=$R_a$, i=$I_a$] (4,1.5)
      (4,1.5) -- (4,0);
  \end{circuitikz}
  \caption{}
  \label{fig:amperimetro_con_r_shunt}
\end{figure}

Para alcances más altos se logran con derivadores o shunts dispuestos de modo tal que por la bobina móvil circule el valor nominal o alcance del instrumento. 

Veamos la deducción de como calcular el valor de la resistencia derivadora para un amperímetro de alcance y resistencia interna conocidos. De esquema en la figura \ref{fig:amperimetro_con_r_shunt} la tensión en bornes del instrumento es:
\[
  U_{AB} = R_s I_s = R_a I_a
\]
Despejando el valor de $R_s$:
\[
  R_s=R_a\frac{I_a}{I-I_a}
\]
Denominado con $n$ al poder multiplicador del shunt a la relación entre la corriente de línea y corriente a fondo de escala -alcance propiamente dicho del instrumento sin derivador-: 
\begin{gather*}
  n=\frac{I}{I_a} \\ 
  R_s = \frac{R_a}{n-1}
\end{gather*}

\subsection{Shunts}

Cuando el alcance no es muy grande (hasta los \qty{50}{\ampere} aproximadamente) los shunts pueden disponerse en el interior del instrumento. Para alcances superiores se los coloca exteriormente y con conductores suplementarios se los conecta al instrumento. Para evitar caídas de tensión excesivas por resistencias de contacto, se utilizan resistencias de cuatro terminales. 

Para instrumentos patrones los shunts deben ser externos para cualquier alcance de corriente, además deben poseer un grado de exactitud compatible con la del instrumento; así por ejemplo si el instrumento es de clase 0,2 el error de la resistencia será de 0,2\% como máximo.

Para alcances bajos se utiliza como material alambre de manganina. Para alcances altos los shunts son de barras de cobre en paralelo para lograr una mejor disipación de calor y ajuste. El ajuste fino se logra practicando pequeños orificios o limaduras laterales en las barras.

La caída de tensión en los bornes de la resistencia shunt conectada con el instrumento es:
\[
  U=I\frac{R_sR_a}{R_s+R_a}=I\frac{R_s}{\frac{R_s}{R_a}+1}
\]
Para alcances superiores a las decenas de amperes $R_s/R_a \ll 1$, de modo que en el denominador el cociente puede despreciarse:
\[
  U=IR_s
\]
de manera tal que una resistencia shunt queda identificada por dos valores característicos: su caída de tensión y su corriente nominal.

Los valores de caídas de tensión están normalizados en 45 60 75 100 150 300\unit{\milli\volt}.

La conexión del shunt con el instrumento, para corrientes elevadas debe apartarse al instrumento de la influencia del campo magnético generado alrededor del conductor cuya intensidad de corriente quiere medirse, debe tenerse cuidado en el dimensionamiento de los cables de unión, pues su resistencia debe ser despreciable frente a la resistencia del instrumento $R_a$, caso contrario produce un error sistemático provocando una deflexión en menos en el instrumento.

\subsubsection*{Potencia de consumo de los shunts}

La potencia de consumo o disipación por efecto Joule en el shunt viene dada aproximadamente por 
\[
  P_s = R_s I^2
\]
cuando el poder multiplicador del shunt $n$ es mucho mayor que 1.

El valor de la corriente de línea en función del alcance del instrumento está dado por:
\[
  I=nI_a
\]
Recordando que $R_s=R_a/(n-1)$, resulta 
\[
  P_s = \frac{n^2}{n-1}R_a I^2
\]
Por la misma consideración anterior $n\gg 1$, por lo que la última expresión se simplifica a 
\[
  P_s\approx nP_a
\]
siendo $P_a$ el consumo propio del instrumento.

En consecuencia la potencia de disipación del shunt es $n$ veces la potencia de consumo del instrumento. Así para un miliamperímetro cuyo consumo nominal es de \qty{1}{\milli\watt}, si se lo usa con un derivador para llegar a medir 20.000 veces su alcance, \qty{1000}{\ampere} (siendo \qty{50}{\milli\ampere} el alcance del instrumento), el shunt tendrá una disipación de \qty{20}{\watt}.

Las bobinas móviles de instrumentos a ser usados con shunts para valores altos de corrientes no están devanadas sobre soportes de aluminio para evitar un amortiguamiento excesivo, debido a la corriente de frenado autoinducida en la bobina móvil.

\subsection{Voltímetros}

El instrumento de imán permanente y bobina móvil como medidor de corriente queda definido por su alcance y resistencia interna, por ejemplo \qty{50}{\micro\ampere} - \qty{5000}{\ohm}. El producto de estos dos valores definen el alcance como voltímetro -para el ejemplo, se transforma en un milivoltímetro de \qty{250}{\milli\volt} a fondo de escala. Para mayores alcances se disponen de resistencias adicionales o multiplicadoras en serie con el instrumento, de manera que el nuevo alcance viene dado por la expresión:
\[
  U=U_{AB}=I(R_a+R_m)
\]
\begin{figure}[!ht]
  \centering
  \begin{circuitikz}
    \draw (0,0) to[short, o-,l=A] (2,0) to[R,l=$R_m$] (2,-1.5) to[rmeterwa,t=V,l=$R_a$] (2,-3) to[short,-o,l=B] (0,-3);
  \end{circuitikz}
\end{figure}
Para determinar $R_m$ para un alcance dado y conociendo los datos $R_a$ e $I$ se plantean las ecuaciones:
\[
  U=U_0+IR_m = U_0 + U_0\frac{R_m}{R_a}
\]
El poder multiplicador de la resistencia adicional queda definido por 
\[
  m=\frac{U}{U_0}
\]
Reemplazando en la expresión de $U$:
\[
  m=1+\frac{R_m}{R_a} \qquad \therefore R_m = R_a (m-1)
\]
La resistencia total $R_m+R_a$ define a $R_v$ la resistencia interna total del instrumento. El cociente entre $Rv/U$ determina una característica importante que diferencia al instrumento con otros de la misma clase. Esta característica es a menudo llamada impropiamente ``sensibilidad del voltímetro \unit{\ohm\per\volt}''. Los valores más comunes son: 1.000, 5.000, 10.000, 25.000, 50.000 \unit{\ohm\per\volt}.

La inversa de los \unit{\ohm\per\volt} nos da el valor de la corriente a fondo de la escala.

En mediciones electrotécnicas las caídas de tensiones son ocasionadas por cargas de bajas resistencias por lo que cualquier instrumento que responda a la característica \unit{\ohm\per\volt} de los valores apuntados resulta admisible. No es así para mediciones electrónicas, ocasionadas por componentes de alto valor resistivo, por lo que un instrumento de más de \qty{25000}{\ohm\per\volt} es lo aconsejable para provocar una menor perturbación en el circuito.

\subsubsection*{Voltímetro para corriente alterna}

Si un instrumento de imán permanente y bobina móvil se conecta a una tensión de corriente alterna, la bobina móvil no provocará ninguna lectura detectable y sólo se observará una vibración del índice para frecuencias bajas. Para evitar esto es necesario modificar la forma de onda. El elemento convertidor no es otra cosa que un rectificador.

Suponiendo que se trata de rectificador ideal y que al conjunto de instrumento-rectificador lo sometemos a una señal de la forma:
\[
  u=U_0\sin{\omega t}
\]
La corriente será:
\[
  i = I_0 \sin(\omega t)
\]
El valor medio de la onda vale:
\begin{gather*}
  i_med = \frac{1}{2\pi}\int_0^\pi I_0 \sin(\omega t) d(\omega t) = \frac{1}{2\pi}I_0 [-\cos(\omega t)]_{0}^\pi \\ 
  i_med = \frac{I_0}{\pi}
\end{gather*}
o bien por ser:
\[
  I_\text{ef} = \frac{I_0}{\sqrt{2}} \qquad \text{entonces} \qquad I_\text{medio}=0.45I_\text{ef}
\]
que es el valor que indica el instrumento con rectificador de media onda.

\subsubsection*{Consumo propio}

Cuando se analizó la cupla motora de este instrumento se dedujo que la misma variaba proporcionalmente con la corriente. En la constante de proporcionalidad se incluía a $B$, es decir que a mayor inducción mayor cupla motora. Con los modernos imanes se puede llegar a \qty{0.3}{\tesla} y aún más, mientras que en los instrumentos de hierro móvil y electrodimámicos se crean campos relativamente bajos de solo \qtyrange{0.01}{0.015}{\tesla}. Esta es la razón por la cual con el instrumento de imán permanente y bobina móvil se pueden construir microamperímetros mientras que con los de hierro móvil y electrodinámicos no se puede bajar de los \qtyrange{15}{20}{\milli\ampere}, pues para valores inferiores el consumo se hace inaceptablemente alto. Esto se debe a que si el valor de $B$ es bajo, para mantener valores adecuadamente altos de la cupla motora es necesario aumentar correspondiente el tamaño de la bobina y sobre todo el número de espiras $N$, lo que determina una mayor longitud del alambre y una disminución de la sección, con lo que se incrementa notablemente la resistencia de la bobina. La consecuencia es un aumento de consumo de los miliamperímetros. Cuando estudiamos al instrumento de IPBM como amperímetro, arribamos a la conclusión que al disponer de una resistencia derivadora para aumentar el alcance, la potencia de consumo de la resistencia shunt era $n$ veces la potencia de consumo propio del instrumento. Ahora utilizando el mismo instrumento como voltímetro, deducimos que también el consumo total será $m$ veces el consumo propio del aparato:
\[
  P_\text{cp}=U_a I_s = R_a I_a^2
\]
La potencia de consumo total:
\[
  P_c = R_mI_a^2 + R_a I_a^2 = R_a(m-1)I_a^2+R_aI_s^2 = mR_aI_a^2 
\]
así,
\[
  P_c = mP_\text{cp}
\]
El consumo es pues proporcional al poder multiplicador $m$. Así si por ejemplo un milivoltímetro de alcance \qty{30}{\milli\volt} tiene un consumo de \qty{0.15}{\milli\watt}; al utilizarlo como voltímetro de \qty{1500}{\volt}. ($m=50.000$) el consumo se ampliará a:
\[
  P_c = \qty{7.5}{\watt}
\]
que es un valor bastante aceptable. Esta es la razón por la cual se pueden usar en instrumentos de imán permanente y bobina móvil poderes multiplicadores tan altos. Compárese con un voltímetro de hierro móvil de solo \qty{600}{\volt} cuyo consumo es de \qty{12}{\watt}.

\subsection{Ohmetro serie}

Un instrumento de imán permanente y bobina móvil de campo radial uniforme en serie con las resistencias $R_i$, $R_1$, $R_2$ y la incógnita $X$ configura básicamente el principio del óhmetro serie:
\begin{figure}[!ht]
  \centering
  \begin{circuitikz}
    \draw (3,0) to[rmeterwa,t=$R$,o-] (0,0) to[battery2,l=$E$] (0,1.5) to[R,l=$R_i$] (0,3) to[R,l=$R_1$] (1.5,3) to[R,l=$R_2$,-o] (3,3);
    \draw (3.1,0.1) to[R,l=X] (3.1,2.9);
  \end{circuitikz}
  \hfill
  \begin{circuitikz}
    \draw (3,0) to[short,o-] (0,0) to[battery2,l=E] (0,3) to[R,l=$R_0$,-o] (3,3);
  \end{circuitikz}
\end{figure}

Existe una relación definida entre la corriente $I_x$ y el valor de $X$. En esta relación también entran los restantes parámetros del circuito: $E$, $R_i$ (resistencia interna de la batería), $R_a$ (resistencia del cuadro móvil del instrumento), $R_1$ (resistencia variable, ajuste del cero de escala) y $R_2$ (resistencia fija). Si todos estos parámetros son fijos la escala del instrumento -por ejemplo microamperímetro- puede calibrarse directamente en unidades de \unit{\ohm}.

Supongamos que contamos con un instrumento de IPBM de \qty{50}{\micro\ampere} -valor usual en los multímetros o tester- a fondo de la escala. Para un valor de $X$ igual a cero, cortocircuitando los terminales de entrada, deben seleccionarse los parámetros del circuito $E$, $R_1$, $R_2$ de manera que la aguja del instrumento marque la máxima deflexión (\qty{50}{\micro\ampere}).

Para $X\to\infty$ (terminales desconectados) la corriente será nula.

Los puntos cardinales (extremos) en la escala en ohms estarán invertidos a los correspondientes de la escala en \unit{\micro\ampere}: el ``cero'' estará a la derecha y el ``infinito'' a la izquierda. 

No obstante er la deflexión proporcional a la corriente, la escala no es uniforme. Para su análisis hallaremos un factor $F$ que dependerá de los parámetros del circuito. 

El valor de la corriente $I_x$ para un valor cualquiera $X$ será:
\[
  I_x = \frac{E}{R_i+R_1+R_2+R_a+X}=\frac{E}{R_0+X}
\]
Con $R_0$ incluimos a todas las resistencias internas del circuito.

Si a la expresión anterior dividimos numerador y denominador por $R_0$
\[
  I_x = \frac{\frac{E}{R_0}}{1+\frac{X}{R_0}} = \frac{I_0}{1+\beta}
\]
siendo la corriente necesaria para la máxima deflexión, al cociente entre la resistencia incógnita y la total interna la denominaremos $\beta$.

La escala del instrumento como óhmetro puede estudiarse en función del factor $F$ definido como $I_x/I_0$ y $\beta$ que es valor fraccional de $X$ comparado con la resistencia de entrada $R_0$ del óhmetro:
\[
  F=\frac{1}{1+\beta}
\]
Esta última expresión es una ecuación universal, ya que es válida con independencia de los valores específicos de los parámetros del circuito: $E$, $R_1$, $R_2$, $R_a$.

Del análisis de la expresión de $F$ se determina que para $\beta=1$ vale $0,5$. Esto indica que para una resistencia incógnita $X$ igual a la interna $R_0$ la deflexión acusa la mitad de escala; justamente entre el cero y este punto llamado punto de medio diseño de la escala, pueden efectuarse lecturas sin dificultad, no así en el resto de la escala donde los trazos se comprimen cada vez más, hasta llegar a infinito. %wtf

\subsection{Ohmetro paralelo}

Por lo visto en el circuito anterior -ohmetro serie- el punto medio de diseño de escala queda determinado por $R_0$, aún haciendo mínimas todas las resistencias, queda como importante la propia del cuadro móvil del instrumento de IPBM -para $\qty{50}{\micro\ampere}/\qty{5000}{\ohm}$- Este valor de por sí solo es excesivo para el campo de mediciones pequeñas por lo que el circuito del óhmetro serie es inadmisible para medir con cierta exactitud resistencias bajas.

La otra configuración circuital que nos permite solucionar el inconveniente apuntado, es la del óhmetro shunt: Se lo denomina óhmetro shunt porque la resistencia incógnita, la $X$ a medir se la conecta en paralelo con el instrumento. Como se ve del circuito de la figura () el aparato necesita de una llave interruptora para desconectar la fuente después de efectuada la medición.

Obsérvese que aquí los puntos cardinales de la escala son distintos respecto a los del óhmetro serie. Cuando la incógnita es cero estamos efectuando un cortocircuito en bornes del instrumento, es decir que la única limitación de la corriente está impuesta por la resistencia $R$. Significa pues que para $X$ cero, el índice coincide -con la llave interruptora cerrada- se obtiene el valor máximo de corriente pues para cualquier otro valor de $X$ la misma actúa como resistencia derivadora.

Para analizar la distribución de la escala, repetimos el mismo razonamiento encarado en el óhmetro serie.

El valor de $I_a$ es el de la corriente circulando cuando tenemos conectado al aparato una $X$:
\[
  I_a = \frac{E}{R+\frac{R_a X}{R_a+X}} \frac{R_a X}{(R_a+X)R_a} = \frac{EX}{X(R+R_a)+RR_a}
\]
El valor máximo de la corriente se obtiene -como dijimos- cuando $X$ es infinito:
\[
  I_\infty=\frac{E}{R+R_a}
\]
Definiendo al factor de distribución de escala $F$:
\[
  F=\frac{I_a}{I_\infty} = \frac{1}{1+\frac{R_\rho}{X}}=\frac{1}{1+\frac{1}{\rho}}
\]

Nuevamente hemos hallado una ecuación universal para cualquier valor de los parámetros. De la representación gráfica de $F$ se ve que para el óhmetro shunt la distribución de escala es idéntica a la del óhmetro serie con la condición de que en aquella se halla invertida (puntos cardinales: cero a izquierda e infinito derecha).

También y en forma similar al óhmetro serie para $\rho=1$ la deflexión acusa la mitad de la escala, definido como el ``punto medio de diseño de la escala''.

La forma de obtener distintas escalas es variado las tensiones de la fuente o bien modificado las resistencias en serie con el instrumento.

El campo de aplicación del óhmetro shunt se extiende desde resistencias muy bajas del orden de los micro-ohms, hasta las unidades, mientras que con el ohmetro serie el límite superior puede llegar al orden de los megaohms.

\subsection{Logómetros}

Para la obtención de distintas magnitudes eléctricas a veces es necesario efectuar el cociente entre otras dos magnitudes. Así si se quiere medir una resistencia conociendo $U$ e $I$ (tensión en bornes y corriente que la circula) se debe efectuar el cociente $U/I$. Por definición diremos que logómetros son aquellos instrumentos susceptibles de medir la relación de dos corrientes, de ahí el nombre de ``cocientímetros'' o bien instrumentos de ``bobinas cruzadas''.

Analicemos el principio de funcionamiento: En presencia de un campo magnético uniforme $B$, se ubican dos bobinas rectangulares y solidarias a un mismo eje y con sus planos formando un ángulo de \qty{90}{\degree}. Las dos bobinas supuestas iguales están recorridas por dos corrientes $i_1$ e $i_2$ en los sentidos indicados en la figura \ref{fig:logometro}.
\begin{figure}[!ht]
  \centering 
  \includegraphics[height=6cm]{chapters/3_unit/media/bin/logometro.png}
  \caption{}
  \label{fig:logometro}
  % La verdad esta figura no la entiendo. Te comento como está compuesta. 
  % Parecen dos barras formando una cruz con un angulo theta, atravesado por un campo magnético B. De las puntas de cada palo la cruz salen fuerzas opuestas... no entiendo.
\end{figure}

Las fuerzas $F_1$ y $F_2$ que actúan sobre los lados activos de las bobinas, perpendiculares a la dirección del campo son:
\[
  F_1 = KBi_1 \qquad \text{y}\qquad F_2=KBi_2
\]
siendo $K$ una constante que depende de la longitud de los lados activos de las bobinas y del número de espiras. Si el plano de la bobina recorrida por la corriente $i$ forma un ángulo $\theta$ con un eje normal al campo y si $a$ es el ancho de las bobinas, las cuplas motoras actuantes serán:
\[
  C_1 = af_1 K'i_1 \sin\theta \qquad \text{y} \qquad C_2 = af_2 K'i_2 \cos\theta
\]
Puesto que las dos cuplas tienen sentido contrario y como las bobinas no están sometidas a ninguna cupla recuperadora, el sistema gira hasta que ambas cuplas sean iguales:
\[
  i_1\sin\theta = i_2 \cos \theta \qquad \therefore \tan\theta = \frac{i_2}{i_1}
\]
de manera que la desviación de las bobinas medida a partir de una cierta posición inicial de referencia, estará directamente vinculada con la relación entre las dos corrientes.

\subsubsection*{Óhmetro de imán permanente y bobinas cruzadas}

Si una de las bobinas se coloca en serie con una resistencia $R$ conocida y la otra bobina en serie con una resistencia $R_x$ a medir, al alimentar los dos circuitos en paralelo con una misma tensión continua $U$ se tendrá:
\[
  \frac{I_2}{I_1}=\frac{R_x}{R}
\]
despreciando las pequeñas resistencias internas de las bobinas.
\[
  R_x = R\tan\theta \quad \therefore \theta \approx R_x
\]
Es decir que la escala del instrumento puede calibrarse directamente en ohm, resultando la indicación independiente de la tensión $U$.

Una manera sencilla de interpretar el funcionamiento de este tipo particular de óhmetro es considerar al sistema conformado por un voltímetro -el campo y una de las bobinas- que mide la tensión en bornes de $R_x$ y por un amperímetro -el campo y la otra bobina- que mide la corriente circulante por la resistencia incógnita.

Desde el punto de vista constructivo no es necesario que el campo sea rectilíneo y uniforme, ni que las dos bobinas estén colocadas en ángulo recto.

Basta solamente una disposición tal que a una rotación del sistema móvil corresponda un aumento de una de las cuplas y una disminución de la otra. Obviamente la ley de respuesta del instrumento no será tan simple como la del caso teórico ya considerado y deberá ser determinada experimentalmente; pero la desviación $\theta$ del sistema móvil será siempre una función unívoca de la relación entre las dos corrientes. Además con la calibración experimental se podrá tener en cuenta las resistencias propias de las bobinas que en una primera aproximación se supuso despreciables y que tienen importancias cuando se miden resistencias pequeñas. Sin embargo la principal aplicación de este tipo de óhmetro está sumamente difundida en las mediciones de resistencias elevadas de aislación de máquinas eléctricas, cables conductores, aisladores, etc.

\begin{figure}[!ht]
  \centering
  \begin{subfigure}[b]{0.48\textwidth}
    \centering
    \includegraphics[width=4cm]{chapters/3_unit/media/bin/ohmetro.png}
    \caption{}
    \label{fig:ohmetro_x}
  \end{subfigure}
  \hfill
  \begin{subfigure}[b]{0.48\textwidth}
    \centering
    \includegraphics[width=5cm]{chapters/3_unit/media/bin/manivela.png}
    \caption{}
    \label{fig:ohmetro_manivela}
  \end{subfigure}
  \caption{}
\end{figure}
En la figura \ref{fig:ohmetro_x} se ha representado el esquema de funcionamiento de este instrumento que recibe el nombre de ``megóhmetro'' o vulgarmente conocido como ``megger''. Puesto que en este caso particular las mediciones deben efectuarse con las tensiones de servicio o mayor según las disposiciones de las normas, se utiliza incorporado al aparato un generador manual o bien un dispositivo electrónico capaz de suministrar tensiones de hasta \qty{5000}{\volt}. La escala, que no es lineal, tiene sus puntos cardinales en cero e infinito, midiendo valores intermedios entre décimas a centenares de megohm. En la figura \ref{fig:ohmetro_manivela} se ha dibujado un generador a manivela, este tipo de megger es el del tipo convencional. Los hay con baterías comunes que con la aplicación de circuitos electrónicos adecuados son capaces de generar tensiones elevadas, aptas para la medición de resistencias de aislación.
% No he entendido bien este apartado la verdad. No se ni que preguntar...

\subsection{Lupa de tensión (con diodo zener)}

Así se denomina al instrumento de IPBM con diodo zener. El diodo zener actúa como un dispositivo intermedio entre la cantidad a medir y el dispositivo final: el instrumento.

Antes de analizar el funcionamiento en conjunto daremos unas nociones elementales respecto al zener.

La característica de los mismos varían respecto del diodo común en lo siguiente: Actuando con polarización inversa circula una corriente muy pequeña -del orden de los microamperes- hasta llegar a llamada ``tensión de ruptura'' o ``tensión zener'' que junto con la disipación, son los dos valores característicos del diodo. Aclaremos que el término ruptura es impropiamente llamado, por cuanto el comportamiento es reproducible siempre que el calentamiento térmico no dañe la estructura cristalina del diodo; esta limitación viene representada por la potencia de disipación, admisible y su corriente máxima inversa nominal. Con esta precaución, el proceso es reversible y repetible con alto grado de exactitud. El circuito utilizado como lupa de tensión es el de la figura \ref{fig:zener_arriba}, para expandir la última parte de la escala y la variante dibujada en la figura \ref{fig:zener_abajo} para ampliar el inicio de la escala.
% estoy más perdido que un niño de primaria leyendo un texto de cálculo. Pido disculpas pero ya no estoy entendiendo lo que el libro me quiere transmitir.
\begin{figure}[!ht]
  \centering
  \begin{subfigure}[b]{0.48\textwidth}
    \centering
    \begin{circuitikz}
      \draw (0,0) to[short,o-] (3,0) to[rmeterwa] (3,1.5) to[R] (3,3) to[zzD*] (1.5,3) to[R,-o] (0,3);
      \draw (1.5,0) to[R,*-*] (1.5,3);
    \end{circuitikz}
    \caption{}
    \label{fig:zener_arriba}
  \end{subfigure}
  \hfill
  \begin{subfigure}[b]{0.48\textwidth}
    \centering
    \begin{circuitikz}
      \draw (0,0) to[short,o-] (3,0) to[rmeterwa] (3,1.5) to[R] (3,3) -- (1.5,3) to[R,-o] (0,3);
      \draw (1.5,0) to[zzD*,*-] (1.5,1.5) to[R,-*]  (1.5,3);
    \end{circuitikz}
    \caption{}
    \label{fig:zener_abajo}
  \end{subfigure}
  \caption{}
\end{figure}

Mientras la tensión que cae sobre la resistencia $R_2$ es inferior a la tensión Zener $U_z$ la resistencia que ofrece el diodo es del orden de las megohm es decir que para el instrumento es casi como si actuara un interruptor abierto. A partir de $U_z$ pequeñas variaciones de tensión repercuten con grandes variaciones de corriente que circulan a través del instrumento. Con una adecuada selección de las resistencias $R_1$ y $R_2$, la escala queda calibrada.

En la disposición circuital de la figura \ref{fig:zener_abajo}, el diodo Zener conectado en paralelo con el instrumento no influye en las indicaciones de éste, siempre que la tensión esté por debajo de $U_z$. Cuando se sobrepasa ésta, el diodo actúa como una resistencia ``shunt'' variable y decreciente con la tensión, de manera que la corriente $I$ se deriva a partir de dicha tensión casi totalmente por la rama que lleva incorporado el diodo. Este tipo de circuito es utilizado para proteger de sobretensiones al instrumento de medida.

\subsection{IPBM con termocupla}

Cuando dos alambres que están compuestos de metales diferentes se unen en ambos extremos y se calienta en una de las puntas, se hace presente una corriente continua que fluye en el circuito. Este efecto recibe el nombre de Seebeck, su descubridor en 1821. Si el circuito de la figura 26, se abre en el centro, la tensión de circuito abierto, llamada ``tensión Seebeck'' es una función de la temperatura del punto de unión de los metales y de la composición de los mismos. Todos los metales que sean distintos exhiben este fenómeno. Las combinaciones más comunes de dos metales que se usan para fabricar termocuplas son las siguientes:
\begin{itemize}
  \item Fe-Constantan: \qty{58.5}{\micro\volt\per\degreeCelsius}
  \item Ni con 10\% de cromo: \qty{39.4}{\micro\volt\per\degreeCelsius}
  \item Platino con 10\% de rodio: \qty{10.3}{\micro\volt\per\degreeCelsius}
\end{itemize}
% Aquí no he puesto la supuesta figura 26. Esta figura es un "circuito" que parecen dos cables de metal distinto unidos, y en un extremo se calientan con una vela. 

Para pequeños cambios de temperatura la tensión de Seebeck es linealmente proporcional a la temperatura:
\[
  e=\alpha T
\]
en realidad, la ley es:
\[
  e=\alpha\Delta t + \beta\Delta t^2
\]
siendo $\alpha$ el coeficiente de Seebeck, es decir, la constante de proporcionalidad.

\subsubsection*{Aplicaciones}

La disposición típica de este instrumento analizado funcionalmente consta de:
\begin{enumerate}
  \item Detector primario: compuesto por un elemento calefactor circulado por la corriente a ser medida.
  \item Termocupla: con su juntura caliente en contacto térmico con el calefactor y con su juntura fría a la temperatura ambiente.
  \item Instrumento: de imán permanente y bobina móvil, actuando como milivoltímetro, cuya ley de respuesta será proporcional a la f.e.m. generada por la termocupla.
\end{enumerate}
Puesto que la f.e.m. es proporcional a la elevación de la temperatura en el elemento calefactor la desviación del instrumento será proporcional a la pérdida:
\[
  P=I^2 R
\]
Por esta razón un instrumento con termocupla tiene escala cuadrática midiendo siempre el valor eficaz de la corriente. Por lo tanto es un instrumento apto para mediciones en corriente continua y en corriente alterna, midiendo valores eficaces independientemente de la forma de onda y de la frecuencia.

\subsubsection*{Elemento calefactor}

Es sabido que la resistencia de un conductor es función de la frecuencia -efecto pelicular o Skin-. Este efecto será menor para conductores muy delgados de materiales de alta resistividad. Por ejemplo para conductores de cobre de \qty{0.025}{\milli\meter} de diámetro, el incremento de resistencia es del 1\% para \qty{20}{\mega\hertz}, mientras que para un conductor de constantan del mismo diámetro el aumento de resistencia es del 0,0015\% para \qty{20}{\mega\hertz}. Se concluye que estos últimos conductores pueden ser usados como elementos calefactores hasta frecuencias muy altas sin que introduzcan errores apreciables (frecuencias de 50 y hasta 80 MHz). Si el diseño es apropiado, su impedancia es casi puramente resistiva, aún para altas frecuencias.

\subsubsection*{Termoelemento}

Está constituido por la combinación del elemento calefactor y la termocupla, en el cual se produce la transformación de la energía térmica en eléctrica. Hay diversas formas constructivas de termoelementos: Una de las formas más simples es la de contacto. La juntura está soldada al elemento calefactor y se halla en contacto eléctrico con él. Tiene la ventaja de una respuesta rápida a los cambios de corrientes producidos en el circuito a medir. En corriente alterna, y para frecuencias superiores a la industrial aparecen efectos capacitivos. Para eliminar esto se separa la termocupla del elemento calefactor, encerrando ambos en una ampolla de vidrio. El termoelemento se usa combinado con un milivoltímetro de baja resistencia, tratando de cumplir con la máxima transferencia de energía, obtenida cuando la resistencia del instrumento iguale aproximadamente a la resistencia interna de la termocupla. El inconveniente principal es su baja capacidad a las sobrecargas -inferior al 50\% para no quemar al elemento calefactor. Se puede llegar a obtener combinado con un buen instrumento una clase final igual a 1 con una frecuencia límite de 50 MHz. Consumo: utilizado como amperímetro en alcance de 100 mA (resistencia del elemento calefactor 1 ohm) el consumo es de 10 mW. Para un alcance de 500 mA el consumo se eleva a 50 mW. De las dos últimas aplicaciones del IPBM, se desprende que la denominada "lupa de tensión" tiene uso en el campo de las mediciones eléctricas -por ejemplo medición de tensión a la salida de un generador-; mientras que el IPBM con termocupla se lo usa en el campo de las mediciones electrónicas en alta frecuencia.

\subsection{Efecto Hall}

En el campo de las mediciones de corriente continua el uso del efecto Hall ha encontrado un gran campo de aplicación. Este fenómeno electromagnético se produce cuando un campo magnético y un conductor plano se disponen perpendicularmente. El generador de Hall se usa para efectuar -de forma relativamente sencilla- medición de corriente, tensión y potencia. Las ventajas de este método son numerosas, especialmente cuando se operan en el campo de corrientes elevadas y al actuar como transductor magnético no es necesaria la inserción directa sobre el circuito a medir. En el campo de las mediciones de tensiones altas el uso del generador de Hall permite efectuar las mismas en forma indirecta, evitando al operador todo peligro debido a las tensiones de contacto.

De física sabemos que al pasar una corriente en sentido longitudinal a través de una placa de material conductor o semiconductor, que está sometida a la acción de un campo magnético normal con el plano de la placa, se establece entre los lados de la placa una diferencia de potencial llamada "tensión de Hall".
\[
  U_{ab} = \frac{K_1 K_h B i}{d}
\]
En la última expresión, $B$ representa el valor de la inducción en Tesla, $i$ es el valor de la corriente de control en amperes, $K_1$ es el valor de una constante que depende de la relación entre el largo y el ancho de la placa y que normalmente asume valores comprendidos entre 0,7 y 0,8 Kh es la constante de Hall cuyo valor dependerá del material utilizado y $d$ el espesor de la placa en milímetros.

De la expresión de la tensión $U_{ab}$ surgen las siguientes aplicaciones:
\begin{enumerate}
\item Aplicaciones basadas en la proporcionalidad entre la tensión y el producto $Bi$.
\item Aplicaciones basadas en la proporcionalidad entre la tensión y la inducción $B$, manteniendo constante la corriente de control $i$.
\item Aplicaciones basadas entre la proporcionalidad entre la tensión y la corriente de control, manteniendo constante la inducción.
\end{enumerate}

Para obtener una tensión de Hall que a igualdad de otras condiciones, resulta elevada, es necesario recurrir a materiales semiconductores tales como el Antimoniuro de Indio y el Arseniuro de Indio.

\subsubsection*{Medición de corriente}

Si indicamos con $I$ el valor de la corriente continua que circula por la línea -y que se desea medir-, con $N$ el número de espiras (para este caso igual a uno), con $\mu$ la permeabilidad en el entrehierro y con $B$ el valor de la inducción en entrehierro, podemos escribir:
\[
  B=\mu N I
\]
por lo que la tensión generada será:
\[
  e=KI
\]
siendo
\[
  K=\frac{K_1 K_h \mu_0 N}{d} i
\]
Las resistencias $R_1$ y $R_2$ colocadas en serie con el circuito de control y de salida respectivamente, tienen la misión de hacer al dispositivo prácticamente independiente de la temperatura.

Tratándose del funcionamiento en corriente continua, el circuito magnético no debe ser necesariamente laminado, y puede estar constituido por un bloque fundido de material magnético. Desde el punto de vista práctico el método ofrece óptimos resultados y particularmente conveniente en el campo de las mediciones de elevadas intensidades de corriente o bien, cuando el circuito está sometido a tensiones altas. En el primer caso, el dispositivo actúa como una pinza amperométrica de las utilizadas en corriente alterna (y que algunas de reciente tecnología incluyen los cristales Hall para la medición en continua) presentando la ventaja de no interrumpir el circuito ni provocar errores de inserción. En el segundo caso la ventaja reside en mantener al operador alejado del conductor en tensión, eliminando así el peligro por contacto accidental.

\subsubsection*{Medición de tensión}

La aplicación del generador Hall en las mediciones de tensiones continuas no es tan frecuente como la descripta para la medición de corriente y tiene interés -como ya se ha dicho- cuando se miden tensiones altas.

El circuito magnético consta de una bobina de $N$ espiras alimentada por la línea cuya tensión se desea medir. La corriente que circula por la bobina creará un campo magnético variable con la tensión. Como en el dispositivo usado para la medición de corriente es necesario que en el campo de funcionamiento no se verifiquen fenómenos de saturación magnética.

\subsubsection*{Medición de potencia}

Se ha citado que en el campo de las aplicaciones están aquellas en que la tensión de Hall es proporcional al producto $Bi$. Esta propiedad resulta de interés cuando se desea medir la potencia en un circuito alimentado con corriente continua.

El campo magnético necesario para el funcionamiento del generador Hall es obtenido por la misma corriente circulante por la línea que constituye la única espira que abraza el núcleo magnético. Si la corriente es débil para la creación de un $B$ útil, será necesario abrazar al núcleo con mayor número de espiras.

La tensión del circuito bajo medida actúa sobre el circuito de control, intercalando una resistencia $R_1$ con funciones limitadoras de corriente. De esta manera la corriente de salida del generador Hall resulta proporcional al producto $UI$, es decir a la potencia.

El grado de exactitud alcanzado con la medición efectuada con este método está relacionado con varios factores como ser la linealidad del circuito magnético, la temperatura, etc., que tomando las precauciones del caso es posible obtener exactitudes próximas al 0,5\%.

\subsection{Errores sistemáticos}

Las causas que dan origen a este tipo de errores pueden resumirse en las siguientes 
\begin{enumerate}
  \item Variación de temperatura
  \item Inestabilidad del imán permanente
  \item Aparición de los efectos termoeléctricos
\end{enumerate}

\subsubsection*{Variación de la temperatura}

La mayoría de los instrumentos indicadores tienen como temperatura de referencia, a la cual fueron calibrados, la llamada temperatura de calibración \qtyrange{20}{25}{\degreeCelsius}, para otros valores la indicación se verá afectada por la incidencia en las siguientes partes constitutivas del aparato:
\begin{description}
  \item[Modificación de la constante motora ($G$)]
Esta es producida en razón de que el aumento de temperatura desmagnetiza el imán permanente reduciendo su flujo y consecuentemente la pérdida de inducción en el entrehierro hace que la constante motora disminuya proporcionalmente. En forma experimental se ha logrado determinar que la reducción del par motor es lineal y con coeficiente porcentual de -0.02\% por cada grado de variación positiva de la temperatura. Este valor porcentual se obtiene considerando la diferencia de deflexión referida a la obtenida para la temperatura de referencia.
  \item[Modificación de la constante elálstica]
    Los muelles en espiral por su característica metálica, si son sometidos al aumento de temperatura reducen su elasticidad en un valor aproximadamente constante en +0.04\%\unit{degreeCelsius} para un determinado rango de temperatura. Esta pérdida de elasticidad es temporal ya que desaparecida la causa el espiral retoma su características originales.
  \item[Variación de la resistencia de la bobina móvil]
    Analicemos este caso en el amperímetro, es decir el cuadro móvil en paralelo con la resistencia shunt. Cuando la conexión del shunt es externa debe tenerse cuidado de efectuar las conexiones como se indicó en el análisis de las características del shunt.Siempre los conductores de linea deben conectarse directamente al shunt y los que unen al instrumento deben ser de dimensiones lo suficientemente cortas para no aumentar la resistencia propia del cuadro móvil, provocando un error de inserción. En condiciones ideales de funcionamiento debe cumplirse para las dos ramas Rs y Ra el mismo coeficiente de temperatura y la misma temperatura de funcionamiento. Para que las dos ramas tengan el mismo coeficiente de temperatura (si la resistencia shunt es de manganina, por ejemplo) se compensa la variación de resistencias agregando en serie con la bobina un resistor de manganina de resistencia r. Analicemos con esta disposición la variación porcentual de resistencia:
    \[
      R=R_a +r 
    \]
    Cuando se produce un aumento de temperatura la $R$ resulta
    \[
      R'=R_a(1+\alpha_{Cu}\Delta t) + r(1+\alpha_{Mn}\Delta t)
    \]
    Para una temperatura de \qty{20}{\degreeCelsius}:
    \[
      \alpha_{Cu} = 4\times 10^{-3} \unit{\ohm\per \degreeCelsius} \quad \text{y} \quad 
      \alpha_{Mn} = 6\times 10^{-6} \unit{\ohm\per \degreeCelsius}
    \]
  Como se observa esta variación es tan poco perceptible frente a la del cobre que la última expresión puede simplificarse:
  \[
    R'=R_a + R_a\alpha_{Cu}+\Delta t +r_{Mn}
  \]
  La variación de resistencia será:
  \[
    \Delta R = R' - R = R_a \alpha_{Cu} \Delta t
  \]
  La variación relativa:
  \[
    \frac{\Delta R}{R} = \frac{R_a}{R_a+r} \alpha_{Cu} \Delta t
  \]
  Si se elige el valor de la resistencia adicional $r = 9R_a$, la variación de resistencia se reduce a un 10\%. Debe tenerse presente que para mantener el mismo poder multiplicador $n$, habrá que aumentar 10 veces el valor de $R_s$ ocasionando un consumo diez veces mayor:
  \[
    n=\frac{R_a + 9R_a}{10R_s}
  \]
  El otro inconveniente es el aumento con esta disposición del error sistemático de inserción al intercalar en la línea una resistencia que perturbara al circuito. En la práctica existe pues, una solución de compromiso adoptándose un valor de $r$ aproximadamente igual a cinco veces el valor de la resistencia del cuadro móvil. Ahora bien antes de analizar esta solución, el efecto total dará una deflexión en defecto por grado centígrado (sumando la variación de la constante elástica $K_r$, la de la constante motora $G$ y la de la resistencia interna $R_a$):
  \begin{gather*}
    \Delta K_r + \Delta G + \Delta r_a = At \\ % Aquí estimo que ha habido un error tipográfico. Sumado a que estoy algo perdido, he vuelto a perderme... 
    +0.04\%/\unit{\degreeCelsius} -0.02 \%/\unit{\degreeCelsius} - 0.4 \%/\unit{\degreeCelsius} = -0.38 \%/\unit{\degreeCelsius}
  \end{gather*}
  Si en cambio se opta por la solución antes citada y adoptando $r=9R_a$ el coeficiente de temperatura pasa a $0.04\%/\unit{\degreeCelsius}$, por lo que la incidecia total se reduce a:
  \[
    +0.04\%/\unit{\degreeCelsius} -0.02 \%/\unit{\degreeCelsius} - 0.04 \%/\unit{\degreeCelsius} = -0.02 \%/\unit{\degreeCelsius}
  \]
\end{description}

\subsubsection*{Estabilidad del imán}

La estabilidad del imán permanente depende entre otras cosas del diseño y tratamiento para su envejecimiento -sometiéndolo a temperaturas de \qty{100}{\degreeCelsius} durante horas-. A pesar de esto el imán con el correr del tiempo sufre una lenta desmagnetización que se traduce en una pérdida de la exactitud del instrumento. Una forma de corregir el problema es actuando sobre el shunt magnético.

\subsubsection*{Influencia de los campos externos}

Si bien la incidencia de los campos extraños no es tan importante como en instrumentos de otro tipo de funcionamiento, la indicación se ve afectada más aun cuando el aparato no posee blindaje. En este caso el mayor error es producido cuando la dirección del campo exterior resulta normal a la dirección de la línea de fuerza simétrica en el entrehierro.

\subsubsection*{Aparición de efectos termoeléctricos}

Esta influencia aparece cuando se efectúan conexiones entre elementos constitutivos de distinto material -como por ejemplo cuando se conecta el cuadro de la bobina móvil, de cobre, con una resistencia adicional de manganina-. Esto trae aparejado los mismos efectos (Seebeck) ya analizados en los instrumentos de IPBM con termocupla. La aparición de esta f.e.m. espurias provocan perturbaciones en las indicaciones tanto más importante cuanto más sensible sea el instrumento.

\subsection{Consideraciones prácticas IPBM con rectificador}

Supongamos a título de ejemplo un multímetro comercial de IPBM que en la función de tensión en continua posee una característica ohm/volt de 20.000. Cuando este mismo instrumento pasa a la función de medir tensiones en corriente alterna su resistencia interna disminuye a 5.000 ohm/volt. Estos valores característicos son fácilmente detectables pues figuran inscriptos en el cuadrante del instrumento.
\begin{figure}[!ht]
  \centering
  \begin{tikzpicture}[>=stealth, scale=1.2]

    % --- Configuración de Ejes ---
    % Eje X (Voltaje)
    \draw[->, thick] (-4,0) -- (4,0) node[right] {$V$ (Voltios)};
    % Eje Y (Corriente)
    \draw[->, thick] (0,-3) -- (0,4) node[above] {$I$ (mA)};

    % --- Curva del Diodo de Germanio ---
    % Usamos el color azul para destacar la curva
    % Polarización Inversa (Breakdown + Fuga)
    \draw[blue, very thick, smooth] 
        (-3.5, -2.8) -- (-3.5, -0.2) % Región de ruptura (Zener/Avalancha)
        .. controls (-3.0, -0.1) and (-1, -0.05) .. (0,0); % Corriente de fuga (pequeña pendiente)

    % Polarización Directa (Exponencial)
    % El codo empieza notablemente antes (0.3V) que un diodo de Silicio
    \draw[blue, very thick, smooth] 
        (0,0) .. controls (0.2, 0.05) and (0.25, 0.1) .. (0.3, 0.5) % Inicio del codo
        .. controls (0.35, 1.5) and (0.4, 3) .. (0.5, 3.8); % Crecimiento exponencial

    % --- Etiquetas y Ayudas Visuales ---
    
    % 1. Voltaje de Codo (0.3V para Germanio)
    \draw[dashed, gray] (0.3, 0) -- (0.3, 0.5);
    \node[below, text=red] at (0.3, 0) {\small $0.3$ V};
    \node[right, text=red, font=\footnotesize] at (0.35, 0.8) {Región de conducción (Ge)};

    % 2. Voltaje de Ruptura
    \draw[dashed, gray] (-3.5, 0) -- (-3.5, -0.2);
    \node[above, text=black] at (-3.5, 0) {\small $V_{BR}$};

    % 3. Corriente de Fuga
    \node[below, font=\footnotesize, text=gray] at (-1.5, -0.1) {Corriente de fuga $I_S$};

    % --- Títulos de Cuadrantes (Opcional) ---
    \node[align=center, font=\small] at (2.5, -1) {Polarización\\Directa};
    \node[align=center, font=\small] at (-2, 2) {Polarización\\Inversa};

  \end{tikzpicture}
  \caption{}
  \label{fig:caracteristica_diodo_germanio}
\end{figure}

Veamos la razón de esta disminución. Tomemos como ejemplo un modelo de multímetro comercial (marca Triplett). Este posee 20.000 ohm/volt en c.c. y 5.000 ohm/volt en c.a. En la figura \ref{fig:caracteristica_diodo_germanio} tenemos representada la característica estática del elemento rectificador: diodo de germanio. En la figura \ref{fig:esquema_triplett} tenemos representado el circuito simplificado del Triplett para el alcance de \qty{3}{\volt} en corriente alterna.
\begin{figure}[!ht]
  \centering
  \begin{tikzpicture}
    % Todos estos esquemas en tikz los he realizado para que puedas visualizar las conexiones del circuito. El libro real tiene esquemas en formato png o svg que no resulta útil extraer.
    % Se que no son los mejores esquemas del segmento pero espero que sean de utilidad.
    \draw (0,-1) to[rmeterwa,l=$5K2$] (0,2) to[D*,l=$D_1$,invert] (0,3) to[R,l=$R_1$] (0,5);
    \draw (0,-.5) -- (-2,-.5) to[R,l=$R_2$] (-2,1.5) to[short,-*] (0,1.5);
    \draw (0,-.5) to[short,*-] (2,-.5) to[D*,l=$D_2$] (2,3) to[short,-*] (0,3);
  \end{tikzpicture}
  \caption{}
  \label{fig:esquema_triplett}
\end{figure}
Antes de seguir el análisis de este circuito conviene destacar la gran importancia que reviste el hecho de trabajar con una corriente directa en el diodo comprendida en la zona lineal de la curva característica del diodo. Las características del instrumento son las siguientes:
\begin{itemize}
  \item Alcance: \qty{50}{\micro\ampere}
  \item Resistencia cuadro móvil:\qty{5000}{\ohm}
\end{itemize}

Simplifiquemos aún más el circuito de la figura \ref{fig:esquema_triplett} con un solo diodo el $D_1$ (descartando $D_2$ y la $R_2$). De esta manera nos queda una configuración circuital serie.

Para obtener los \qty{250}{\milli\volt} de tensión media en bornes del instrumento debería circular 50 microamperes con lo cual $R_1$ debería aumentar considerablemente. Pero aún así, para una corriente tan baja entramos en la curva inicial de la característica del diodo en una zona donde no hay conducción posible. Obsérvese que recién para una corriente del orden de la décima del miliampere y una caída de tensión de \qty{0.15}{\volt} entramos recién en la zona de conducción con características aproximadamente lineal. Esto de por si constituye una razón más que suficiente para comprender la imposibilidad de medir tensiones alternas de valores muy bajos.

La resistencia $R_2$ de valor muy bajo (\qty{3}{\ohm}) tiene por misión hacer derivar la intensidad de corriente que excede al alcance del microamperímetro, es decir su función es similar a la resistencia shunt ya estudiada, actuando el instrumento como milivoltímetro. Aún resta el análisis de la inclusión del diodo $D_2$. La necesidad de este diodo se debe a que en el circuito con un solo rectificador, durante el semiciclo con polarización inversa su resistencia a la corriente es grande por lo que soporta la tensión completa del circuito, lo que conduciría a la eventualidad de una perforación, ya que los diodos de germanio la tensión inversa de ``ruptura'' es pequeña.-

\subsection{Características escala en C.A.}

La escala de este instrumento viene calibrada directamente en valores eficaces basados en la suposición de medir ondas sinusoidales es decir que a las divisiones de escala se incorporan el factor de forma de una onda sinusoidal igual a 1,11. Si la forma de onda se aparta de la sinusoidal aparecen errores en las lecturas. Analicemos el problema tomando la ley de deflexión y considerando el factor de forma 1,11 por el que se multiplica a las divisiones de la lectura:
\[
  I_m = \alpha \frac{1.11}{G}
\]
$I_m$ es el valor medido y $G$ la constante motora.

Si el multímetro tuviera un rectificador puente y si la tensión aplicada responde a una sinusoide de valor pico $I_p$, el valor medio de la corriente será:
\[
  I=2\frac{I_p}{\pi}
\]
Como
\[
  \alpha= GI=G 2 \frac{I_p}{\pi}
\]
Finalmente el valor de la corriente medida será:
\[
  I_m = 0.707 I_p
\]
Para este caso el resultado es correcto, la lectura es igual al valor eficaz y es porque así se ha diseñado el instrumento. Pero si la corriente no es sinusoidal podemos tener errores de lectura superiores o inferiores al valor eficaz, dependiendo de la forma de onda que estamos midiendo. Por ejemplo si tenemos una corriente continua $i(t) = I$, la corriente medida será:
\[
  I_m = 1.11  I
\]
Esto significa que tenemos un error en exceso del 11\%. Otro ejemplo lo constituye la onda triangular cuyo valor medio es la mitad del valor pico y el eficaz es $1/\sqrt{3}$ del valor pico. El valor medido será:
\[
  I_m = 1.11 \frac{I_p}{2} = 0.555 I_p
\]
Es decir aproximadamente el 4\% más bajo que el valor eficaz de la onda triangular.

\subsection{Errores por envejecimiento}

Todo óhmetro tiene lo que se denomina ``ajuste de cero'' que no es otra cosa que un resistor variable a efectos de hacer coincidir la aguja con la indicación cero ohm previo cortocircuito de los terminales. Con el desgaste de la batería incorporada, disminuye su tensión en bornes y consecuentemente aumenta su resistencia interna. Es evidente que a medida que la tensión baja irá disminuyendo la resistencia de ajuste de cero. En todos los casos la aguja coincidirá con el cero, a menos que la batería se envejezca tanto que ya sea imposible hacer llegar a fondo de la escala el índice del instrumento. Vamos a demostrar con un ejemplo que una variación excesiva de la tensión de la batería provoca un error de lectura -que se suma al de la exactitud propia del instrumento-, aún cuando hagamos coincidir al cero el índice del óhmetro. En el diseño del óhmetro de la figura se ha utilizado un miliamperímetro de resistencia interna de 50 ohm, de alcance 1 mA, una resistencia fija de 3.000 ohm y la variable ajuste de cero de 2.000 ohm. La batería seccionada es de 4.5 V. Con estos parámetros la resistencia de diseño de punto medio de escala es de 4.500 ohm. Con la batería nueva el valor de la corriente será a fondo de escala:
\[
  I_0 = \frac{\qty{4.5}{\volt}}{\qty{4500}{\ohm}}=\qty{1}{\milli\ampere}
\]
Al conectar en los terminales una resistencia de valor conocido $X=\qty{5000}{\ohm}$ la corriente que circula por el instrumento será:
\[
  I_x = \frac{\qty{4.5}{\volt}}{(4500+5000)\unit{\ohm}}=\qty{0.47}{\milli\ampere}
\]
Ahora analicemos que pasa ante un desgaste de la batería, cuya tensión baja a \qty{4}{\volt} con un incremento de su resistencia interna, el nuevo valor de $R_0$ hará obtener el valor final de \qty{1}{\milli\ampere}.
\[
  I_0 =\frac{\qty{4}{\volt}}{\qty{4000}{\ohm}} = \qty{1}{\milli\ampere}
\]
Cuando se conecte la resistencia de \qty{5000}{\ohm}, el valor de la corriente en el miliamperímetro será:
\[
  I_{x1}=\frac{\qty{4}{\volt}}{(4000+5000)\unit{\ohm}}=\qty{0.44}{\milli\ampere}
\]
Esta disminución de la corriente provoca un error de lectura, que será tanto mayor cuanto menor sea la deflexión angular del cuadro móvil.

El valor que indicará la escala del óhmetro para la corriente $I_{x1}$ será:
\[
  F=\frac{I_{x1}}{I_0}=0.44=\frac{1}{1+\frac{X_m}{4500}}\quad \therefore X_m=\qty{5700}{\ohm}
\]
cometiéndose un error del 14\%.

Para tratar de minimizar este error, algunos óhmetros llevan una resistencia de ajuste en paralelo con el instrumento de manera tal que se logra una menor variación de $R_0$, pues el instrumento es más sensible a la variación de $R_s$.
