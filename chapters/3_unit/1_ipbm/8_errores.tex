\section{Fuentes de Error en Instrumentos IPBM}

Como todo sistema físico, el instrumento de Bobina Móvil no es perfecto. Existen perturbaciones externas e internas que alejan la lectura del valor verdadero. Las clasificamos en tres grupos principales: Temperatura, Envejecimiento y Campos Externos.

\subsection{Influencia de la Temperatura}

Este es el factor más crítico. Los instrumentos suelen calibrarse a una temperatura de referencia (típicamente \qtyrange{20}{25}{\degreeCelsius}). Si operamos fuera de ese rango, el calor afecta a tres componentes clave, produciendo efectos contrapuestos:

\subsubsection{A. Efectos Mecánicos y Magnéticos}
Aquí ocurre una ``lucha'' entre el imán y el resorte:

\begin{enumerate}
    \item El Imán ($B$): Al aumentar la temperatura, el imán pierde fuerza (se desordena su estructura interna). Esto reduce la Cupla Motora.
    \item El Resorte ($K_r$): Al calentarse, el metal pierde elasticidad (se vuelve más blando). Esto reduce la Cupla Antagónica.
\end{enumerate}

Balance:
\begin{itemize}
    \item El imán se debilita aprox. $-0.02\%/\unit{\degreeCelsius}$ (La aguja quiere marcar menos).
    \item El resorte se ablanda aprox. $0.04\%/\unit{\degreeCelsius}$ (La aguja quiere marcar más, porque el freno es menor).
\end{itemize}
El efecto neto mecánico suele ser positivo: el instrumento tiende a marcar ligeramente de más ($\approx +0.02\%/\unit{\degreeCelsius}$).

\subsubsection{B. Efecto Eléctrico (Resistencia de la Bobina)}
Este es el problema grave. La bobina es de Cobre, cuyo coeficiente de temperatura es muy alto ($\alpha_{Cu} \approx 0.4\%/\unit{\degreeCelsius}$).
Si la temperatura sube, la resistencia de la bobina ($R_a$) aumenta drásticamente.

El Problema en Amperímetros (con Shunt):
Si tenemos una bobina de cobre en paralelo con un Shunt de Manganina (que no varía con la temperatura), y la bobina se calienta, su resistencia sube. La corriente, que es ``perezosa'', preferirá irse por el Shunt en lugar de por la bobina.
Resultado: La corriente por el instrumento disminuye drásticamente, provocando un gran error negativo (lee mucho menos de lo real).

Cálculo del Error Total (Sin compensar):
\[ E_{total} = (+0.02\% \text{ Mecánico}) + (-0.4\% \text{ Eléctrico}) \approx \mathbf{-0.38\%/\unit{\degreeCelsius}} \]
Este error es inaceptable para un instrumento de precisión.

\subsubsection{C. Solución: Resistencia de Compensación (Swamping Resistor)}

Para solucionar el problema del cobre, se conecta una resistencia fija de Manganina en serie con la bobina móvil.

\begin{figure}[!ht]
  \centering
  \begin{circuitikz}
    \draw (0,0) to[short, o-] (1,0) -- (1,2)
    to[R, l=$R_\text{comp}$] (3,2)
    to[rmeterwa, t=A, l=$R_\text{bobina}$] (5,2) -- (5,0)
    to[short, -o] (6,0);
    \draw (1,0) to[R, l=$R_\text{shunt}$] (5,0);
  \end{circuitikz}
  \caption{Compensación de temperatura mediante resistencia serie de Manganina.}
  \label{fig:compensacion_temp}
\end{figure}

La idea es que la resistencia de Manganina ($R_\text{comp}$) sea mucho mayor que la de la bobina ($R_\text{bobina}$), por ejemplo, unas 9 veces mayor.
\[ R_\text{total} = R_\text{bobina} + R_\text{comp} \]

Al aumentar la temperatura:
\begin{itemize}
    \item $R_\text{bobina}$ sube un 40\% (ejemplo exagerado).
    \item $R_\text{comp}$ no cambia (0\%).
\end{itemize}
Como $R_\text{comp}$ es la parte mayoritaria de la resistencia total, el cambio porcentual del conjunto se diluye. Matemáticamente, el coeficiente de temperatura efectivo baja de $0.4\%$ a apenas $0.04\%$.

El costo de esta solución: Al agregar resistencia extra en serie, aumentamos la caída de tensión total del instrumento. Perdemos sensibilidad o ``cargamos'' más el circuito (Error de inserción), pero ganamos estabilidad térmica. Es una solución de compromiso.

\subsection{Inestabilidad del Imán (Envejecimiento)}

Con el paso de los años, los imanes permanentes tienden a relajarse y perder magnetismo de forma natural, lo que haría que el instrumento marque cada vez menos.
\begin{itemize}
    \item Prevención: En fábrica, los imanes se someten a un ``envejecimiento artificial'' (ciclos de calor y campos desmagnetizantes) para estabilizarlos antes de armar el instrumento.
    \item Corrección: Si el error aparece con los años, se corrige ajustando el derivador magnético (ese tornillo de bypass que vimos al principio) para enviar más flujo a la bobina.
\end{itemize}

\subsection{Efectos Termoeléctricos (Parásitos)}

Dentro del instrumento hay uniones de metales distintos (ej. la bobina de Cobre soldada a los resortes de Bronce o a resistencias de Manganina).
Si existe un gradiente de temperatura dentro de la carcasa (un lado más caliente que otro), estas uniones actúan como termocuplas indeseadas, generando pequeñas tensiones (f.e.m. de Seebeck) que se suman o restan a la medición real.
Solución: Diseño simétrico y evitar fuentes de calor cercanas a los terminales.

\subsection{Campos Magnéticos Externos}

Aunque el instrumento tiene su propio campo interno fuerte, un campo externo potente (transformadores, cables de alta corriente cercanos) puede sumar o restar fuerza a la aguja.
Solución: Blindaje. El mecanismo se encierra en una carcasa de hierro dulce o acero que actúa como jaula magnética, desviando las líneas de campo externas por las paredes de la carcasa sin que toquen la bobina.
