\section{Profundización en Medición de C.A.: Rectificadores}

Como mencionamos anteriormente, el instrumento PMMC es intrínsecamente de corriente continua. Para medir alterna, utilizamos rectificadores. Sin embargo, esto introduce limitaciones prácticas y fuentes de error que no existen en CC.

\subsection{Pérdida de Sensibilidad y No Linealidad}

Un hecho observable en los multímetros analógicos comerciales es la drástica disminución de su sensibilidad al cambiar de modo CC a CA. Por ejemplo, un equipo que presenta \qty{20000}{\ohm\per\volt} en continua, típicamente baja a solo \qty{5000}{\ohm\per\volt} en alterna. ¿A qué se debe esto?

La causa principal reside en la naturaleza física de los diodos rectificadores (generalmente de Germanio en instrumentos antiguos o Schottky en modernos).

\begin{figure}[!ht]
  \centering
  \begin{tikzpicture}[>=stealth, scale=1.0]
    % Ejes
    \draw[->, thick] (-4,0) -- (4,0) node[right] {$V$ [V]};
    \draw[->, thick] (0,-3) -- (0,4) node[above] {$I$ [mA]};
    
    % Curva característica del Germanio
    % Zona inversa (fuga y ruptura)
    \draw[blue, very thick, smooth] 
        (-3.5, -2.8) -- (-3.5, -0.2) 
        .. controls (-3.0, -0.1) and (-1, -0.05) .. (0,0);
    
    % Zona directa
    \draw[blue, very thick, smooth] 
        (0,0) .. controls (0.2, 0.05) and (0.25, 0.1) .. (0.3, 0.5) 
        .. controls (0.35, 1.5) and (0.4, 3) .. (0.5, 3.8);

    % Cotas y textos
    \draw[dashed, gray] (0.3, 0) -- (0.3, 0.5);
    \node[below, text=red] at (0.3, 0) {\small $0.3$ V};
    \node[right, text=red, font=\footnotesize] at (0.4, 1) {Zona de conducción};
    \node[left, text=black] at (-3.5, -1) {$V_{ruptura}$};
  \end{tikzpicture}
  \caption{Característica V-I de un diodo de Germanio. Nótese el umbral de conducción en 0.3V.}
  \label{fig:curva_diodo}
\end{figure}

Para que el instrumento mida linealmente, el diodo debe trabajar en su zona de conducción recta. Sin embargo, como vemos en la figura \ref{fig:curva_diodo}, los diodos necesitan una tensión umbral mínima para empezar a conducir (aprox. \qty{0.3}{\volt} para Germanio).

Si intentamos medir tensiones alternas muy bajas (ej. \qty{250}{\milli\volt}), la señal no tendrá fuerza suficiente para ``abrir'' el diodo. La corriente resultante será casi nula o extremadamente no lineal. Esto hace imposible fabricar voltímetros de alterna de rangos muy bajos con esta tecnología simple.

\subsubsection{Análisis del Circuito Rectificador}

Analicemos un circuito comercial típico (Triplett) simplificado para bajos voltajes de CA:

\begin{figure}[!ht]
  \centering
  \begin{tikzpicture}[american voltages]
    % Circuito simplificado
    \draw (0,0) to[short, o-] (1,0) 
    to[R, l=$R_\text{cal}$] (3,0) -- (4,0);
    
    % Rama del instrumento
    \draw (4,0) to[D*, l=$D_1$, i=$I_m$] (4,2) 
    to[rmeterwa, l=$PMMC$] (4,3.5) -- (1,3.5) 
    to[short, -o] (0,3.5);
    
    % Diodo de protección D2 en antiparalelo
    \draw (4,0) -- (5.5,0) 
    to[D*, l=$D_2$] (5.5,3.5) -- (4,3.5);
    
    % Resistencia shunt R2
    \draw (2.5,0) to[R, l=$R_\text{shunt}$] (2.5,3.5);
  \end{tikzpicture}
  \caption{Esquema de rectificación de media onda con protección.}
  \label{fig:circuito_rectificador}
\end{figure}

El circuito emplea dos diodos con funciones muy distintas:
\begin{itemize}
    \item Diodo $D_1$ (Rectificador): Es el encargado de dejar pasar los semiciclos positivos hacia el instrumento para producir la deflexión.
    \item Diodo $D_2$ (Protección): Durante el semiciclo negativo, $D_1$ bloquea la corriente. Toda la tensión inversa de la red caería sobre $D_1$, pudiendo perforarlo si supera su tensión de ruptura (que es baja en Germanio). Para evitarlo, se coloca $D_2$ en antiparalelo. En el semiciclo negativo, $D_2$ conduce y cortocircuita la señal, protegiendo a $D_1$ y garantizando la continuidad del circuito de CA.
\end{itemize}

\subsection{El Factor de Forma y Errores de Onda}

Esta es una fuente de error conceptual muy común. El instrumento PMMC responde, por principios físicos, al Valor Medio de la corriente rectificada.
Sin embargo, al usuario le interesa conocer el Valor Eficaz (RMS).

Para solucionar esto, los fabricantes diseñan la escala ``mintiendo'': dibujan los números multiplicados por el Factor de Forma de una onda senoidal pura ($K_f = 1.11$).
\[ K_f = \frac{Valor\_Eficaz}{Valor\_Medio} = \frac{\pi}{2\sqrt{2}} \approx 1.11 \]

Consecuencia: La lectura solo es correcta si medimos ondas senoidales puras.
Si medimos otra forma de onda, el instrumento aplicará erróneamente el factor 1.11 al valor medio real de esa onda, provocando errores significativos:
\begin{itemize}
    \item Onda Cuadrada: El instrumento marcará un 11\% en exceso.
    \item Onda Triangular: El instrumento marcará un 4\% en defecto aproximadamente.
\end{itemize}
