\section{Medición de Resistencia (Óhmetro)}

A diferencia de los amperímetros y voltímetros, que miden magnitudes generadas por un circuito externo, el óhmetro debe proporcionar su propia energía para realizar la medición. Básicamente, su principio consiste en aplicar una tensión conocida (batería interna) a la resistencia incógnita y medir la corriente resultante. Según la Ley de Ohm, la corriente será función de la resistencia.

Existen dos configuraciones básicas: el Óhmetro Serie (para resistencias medias y altas) y el Óhmetro Shunt (para resistencias muy bajas).

\subsection{Óhmetro Serie}

Es la configuración más común, utilizada en la mayoría de los multímetros analógicos universales. Se denomina ``serie'' porque la batería, el instrumento de medida y la resistencia incógnita ($R_x$) se conectan en cadena (o en serie).

\begin{figure}[!ht]
  \centering
  \begin{circuitikz}[american voltages]
    \draw (0,0) to[battery1, l=$E$] (0,2)
      to[vR, l=$R_\text{aj}$] (2,2) % Resistencia de ajuste de cero
      to[rmeterwa, t=A, l=$R_b$] (4,2)
      to[short, -o] (5,2) node[right=.5cm] {\scriptsize{Terminal +}};
    \draw (0,0) to[R, l=$R_f$] (2,0) % Resistencia fija limitadora
      to[short, -o] (5,0) node[right=.5cm] {\scriptsize{Terminal -}};
    \draw (5.2, 2) to[R, l=$R_x$] (5.2, 0);
  \end{circuitikz}
  \caption{Circuito básico de un Óhmetro Serie. $R_\text{aj}$ permite el ajuste de puesta a cero.}
  \label{fig:ohmetro_serie}
\end{figure}

Para analizar su funcionamiento, agrupamos todas las resistencias internas del aparato ($R_\text{batería} + R_\text{ajuste} + R_\text{bobina} + R_\text{fija}$) en una única resistencia total interna a la que llamaremos $R_i$.

La corriente que circula por el instrumento al conectar una incógnita $R_x$ es:
\begin{equation}
  I = \frac{E}{R_i + R_x}
\end{equation}

Analicemos los casos extremos para entender la escala:
\begin{enumerate}
    \item \textbf{Cortocircuito ($R_x = 0$):} Si unimos las puntas de prueba, la corriente es máxima.
    \[ I_\text{max} = \frac{E}{R_i} \]
    En este punto, ajustamos $R_\text{aj}$ (potenciómetro de ``Ajuste de Cero'') para que la aguja llegue exactamente al fondo de la escala. Por tanto, el \textit{Fondo de Escala corresponde a \qty{0}{\ohm}}.

    \item \textbf{Circuito Abierto ($R_x \to \infty$):} Si las puntas están separadas, la corriente es nula ($I=0$). La aguja no se mueve. Por tanto, el \textit{Inicio de la Escala (izquierda) corresponde a $\infty$}.
\end{enumerate}

\textbf{Conclusión:} La escala del óhmetro serie es \textbf{inversa}. El cero está a la derecha y el infinito a la izquierda.

\subsubsection{Forma de la Escala y el Punto Medio}

Si dividimos la ecuación general por la corriente máxima ($I_\text{max} = E/R_i$), obtenemos la deflexión relativa $F$:
\[
  F = \frac{I}{I_\text{max}} = \frac{\frac{E}{R_i+R_x}}{\frac{E}{R_i}} = \frac{R_i}{R_i + R_x} = \frac{1}{1 + \frac{R_x}{R_i}}
\]
Llamando $\beta$ a la relación $R_x/R_i$, tenemos la ecuación universal de la escala:
\[ F = \frac{1}{1 + \beta} \]

¿Qué sucede cuando la resistencia incógnita es igual a la resistencia interna ($R_x = R_i$ o $\beta=1$)?
\[ F = \frac{1}{1+1} = 0.5 \]
Esto significa que \textbf{en el centro geométrico de la escala (50\% de deflexión) leemos el valor de la resistencia interna del instrumento}.

\textbf{¿Por qué se comprime la escala?}
La relación entre corriente y resistencia es hiperbólica (del tipo $y=1/x$).
\begin{itemize}
    \item Para valores bajos de $R_x$, pequeños cambios de resistencia provocan grandes cambios de corriente (escala expandida y legible).
    \item Para valores muy altos de $R_x$, grandes cambios de resistencia apenas afectan a la corriente total (que ya es muy pequeña). Esto provoca que las divisiones se ``amontonen'' o compriman en la zona izquierda de la escala, haciendo imposible la lectura precisa cerca del infinito.
\end{itemize}

\subsection{Error por Envejecimiento de Batería (Óhmetro)}

Una consideración crítica en el diseño del óhmetro serie es la dependencia de su fuente de energía interna. Con el uso, la batería se desgasta: su fuerza electromotriz ($E$) disminuye y su resistencia interna aumenta.

El óhmetro dispone de un potenciómetro de ``Ajuste de Cero'' ($R_\text{aj}$) que permite compensar esto. Antes de medir, unimos las puntas (cortocircuito) y variamos $R_\text{aj}$ hasta que la aguja marque $0\,\Omega$ (fondo de escala de corriente).
Al bajar la tensión de la batería, debemos reducir $R_\text{aj}$ para dejar pasar más corriente y alcanzar nuevamente el fondo de escala.

\textbf{El Problema del Centro de Escala}
Aunque logremos ajustar el cero, hemos alterado la resistencia interna total del instrumento ($R_i$). Recordemos que el valor central de la escala ocurre cuando $R_x = R_i$.

Si $R_i$ disminuye (porque bajamos $R_\text{aj}$ para compensar la batería), el centro de la escala se desplaza.

\textbf{Ejemplo Numérico:}
Supongamos un óhmetro diseñado para $E=\qty{4.5}{\volt}$ con una resistencia central de diseño de \qty{4500}{\ohm}.
\[ I_\text{max} = \frac{4.5}{4500} = \qty{1}{\milli\ampere} \]
Si medimos una resistencia incógnita real de $R_x = \qty{5000}{\ohm}$, la corriente será:
\[ I_{x} = \frac{4.5}{4500 + 5000} \approx \qty{0.47}{\milli\ampere} \]

Ahora, supongamos que la batería cae a $E'=\qty{4.0}{\volt}$. Para ajustar el cero, reducimos la resistencia interna a \qty{4000}{\ohm} ($I = 4/4000 = 1$ mA). El cero parece correcto.
Pero al medir la misma incógnita de \qty{5000}{\ohm}:
\[ I'_{x} = \frac{4.0}{4000 + 5000} \approx \qty{0.44}{\milli\ampere} \]

El instrumento leerá esta corriente menor como si la resistencia fuera mayor. Aplicando la ecuación de escala inversa, la aguja indicará aproximadamente \qty{5700}{\ohm} en lugar de \qty{5000}{\ohm}.
Esto representa un error del 14\%, causado únicamente por el agotamiento de la batería, a pesar de haber realizado el ajuste de cero.

\subsection{Óhmetro Shunt (o Paralelo)}

El óhmetro serie tiene una limitación: su resistencia interna suele ser de miles de Ohms. Si intentamos medir una resistencia muy pequeña (ej. \qty{0.1}{\ohm}), la variación de corriente será imperceptible.
Para medir bajas resistencias, se utiliza la configuración \textbf{Shunt}.

\begin{figure}[!ht]
  \centering
  \begin{circuitikz}[american voltages]
    \draw (0,0) to[battery1, l=$E$] (0,2)
      to[push button, l=S, -o] (1.5,2) % Interruptor pulsador
      to[R, l=$R_s$] (3.5,2) -- (5,2);
    \draw (0,0) -- (5,0);
    % Rama del instrumento
    \draw (3.5,2) to[rmeterwa, t=V, l=$R_v$] (3.5,0);
    % Bornes de prueba
    \draw (5,2) to[short, -o] (6,2);
    \draw (5,0) to[short, -o] (6,0);
    \draw (6,2) to[R, l=$R_x$] (6,0);
  \end{circuitikz}
  \caption{Óhmetro Shunt. La incógnita $R_x$ se conecta en paralelo al voltímetro.}
  \label{fig:ohmetro_shunt}
\end{figure}

Aquí, el instrumento actúa esencialmente como un voltímetro que mide la caída de tensión sobre la resistencia incógnita $R_x$. El circuito consta de una fuente $E$ y una resistencia limitadora serie $R_s$, y la incógnita se conecta en \textbf{paralelo} al instrumento.
Es fundamental incluir un interruptor (normalmente un pulsador) para no descargar la batería, ya que al no medir nada ($R_x = \infty$), sin embargo el circuito consume parte de la corriente que circula por el voltímetro, que mide la tensión de la batería.

Analicemos los extremos:
\begin{enumerate}
    \item \textbf{Cortocircuito ($R_x = 0$):} La corriente se desvía totalmente por la incógnita. No hay tensión sobre el instrumento. La aguja marca cero. Aquí \textbf{el 0 está a la izquierda}.
    \item \textbf{Circuito Abierto ($R_x \to \infty$):} Toda la corriente pasa por el instrumento. La aguja deflecta al máximo. Aquí \textbf{el $\infty$ está a la derecha}.
\end{enumerate}

\textbf{Conclusión:} La escala del óhmetro shunt tiene el sentido ``normal'' (crece de izquierda a derecha), al contrario que el serie.

\subsubsection{Ecuación de la Escala}
La tensión que mide el instrumento depende de $R_x$. Considerando $R_p$ como la resistencia paralelo formada por el instrumento y la incógnita:
\[ U_\text{medido} = E \cdot \frac{R_p}{R_s + R_p} \quad \text{donde } R_p = \frac{R_v \cdot R_x}{R_v + R_x} \]

Para simplificar el análisis conceptual: el punto de media escala ($F=0.5$) ocurre cuando la resistencia incógnita $R_x$ es igual a la resistencia equivalente que ``se ve'' desde sus bornes (Thevenin), que es aproximadamente la resistencia interna del instrumento en paralelo con la serie.

Este instrumento es ideal para medir resistencias muy bajas (desde micro-ohms hasta algunos ohms), complementando perfectamente al óhmetro serie.
