\chapter{Instrumentos electrodinámicos}

Los instrumentos electrodinámicos están constituidos por dos bobinas cruzadas a \qty{90}{\degree}; una de las cuales es fija y la otra es móvil. La cupla motriz es producida por la interacción de los campos magnéticos producidos en ambas bobinas cuando por ellas circula una corriente.

Estos instrumentos podrían utilizarse como amperímetros o voltímetros conectando ambas bobinas en serie y los shunt convenientes. Pero sin embargo su utilidad más importante es como \emph{vatímetro} en la medición de la potencia eléctrica $P$ (Potencia Activa).

En el caso del vatímetro, la bobina fija de baja resistencia interna, se conecta en serie con la carga para que mida la corriente, y se denomina bobina amperométrica. La bobina móvil de elevada resistencia interna, se conecta en paralelo con la carga de tal manera que mida la tensión, y se denomina la bobina voltimétrica.

La cupla amortiguante es provocada por una paleta o émbolo.

\section{Ley de deflexión}

El campo magnético formado por la bobina fija es, en todo momento, proporcional al valor instantáneo de la intensidad que la recorre: \(i_A \sim B\). Así mismo, la intensidad en la bobina móvil es proporcional a la tensión \(i_u \sim u\)

En la bobina móvil aparece una cupla cuyos valores instantáneos $C_m$ son proporcionales al producto de la inducción y la intensidad
\[
  C_m \sim B\cdot i_u \sim i_A \cdot u \sim p
\]
que es proporcional al producto de la intensidad \(i_A\) en la bobina fija y la tensión \(u\) aplicada a la bobina móvil, es decir, a la potencia instantánea \(p\).

Como la potencia activa \(P\) es el valor medio de las potencias instantáneas \(p\), resulta que
\[
  P_\text{med} = P=V\cdot I\cdot \cos(\varphi) \sim (C_m)_\text{med}
\]
El sistema móvil del instrumento, por su inercia, se comporta como un convertidor de cupla: La cupla entrante, alternativa, asimétrica $C_m$ queda convertida en una cupla de salida de valor constante $(C_m)_\text{med}$ que se aplica sobre los resortes, haciéndolos ceder el ángulo $\alpha$. Llamando $C_d$ a la cupla de los resortes se tiene que:
\[
  (C_m)_\text{med} = C_d \sim \alpha
\]
\begin{figure}[!ht]
  \centering
  \includegraphics[width=0.8\textwidth]{chapters/3_unit/media/bin/electrodinamico.pdf}
  \caption{Instrumento electrodinámico.}
\end{figure}

\section{Análisis de la Cupla y Ley de Deflexión}

Para entender el movimiento, partimos de la energía magnética almacenada en el sistema de dos bobinas. La cupla motriz instantánea \(C_m(t)\) que tiende a girar la bobina móvil está dada por la variación de la energía magnética respecto al ángulo de giro \(\alpha\):

\begin{equation}
    C_m(t) = i_f(t) \cdot i_m(t) \cdot \frac{dM}{d\alpha}
\end{equation}

Donde:
\begin{itemize}
    \item \(i_f, i_m\): Corrientes instantáneas en bobina fija y móvil.
    \item \(M\): Inductancia mutua entre las bobinas (depende de la posición \(\alpha\)).
\end{itemize}

Esta cupla motriz se equilibra con la cupla antagonista de los resortes \(C_r = k \cdot \alpha\). Analicemos cómo resulta la deflexión final \(\alpha\) en cada régimen.

\subsection{Caso 1: Corriente Continua (CC)}
En corriente continua, las intensidades son constantes (\(I_F\) e \(I_M\)). La cupla motriz es constante y la aguja se desplaza hasta que se iguala con el resorte:

\[
    C_m = I_F \cdot I_M \cdot \frac{dM}{d\alpha} = k \cdot \alpha
\]

Por lo tanto, la \textbf{ley de deflexión para CC} es:
\begin{equation}
    \alpha = \frac{I_F \cdot I_M}{k} \cdot \frac{dM}{d\alpha}
\end{equation}

\textit{Nota conceptual:} Si se usa como amperímetro (las dos bobinas en serie, \(I_F=I_M=I\)), la escala es cuadrática (\(\alpha \sim I^2\)). Si se usa como vatímetro, es lineal respecto a la potencia.

\subsection{Caso 2: Corriente Alterna (CA)}
En alterna, las corrientes varían sinusoidalmente con el tiempo. Supongamos que están desfasadas un ángulo \(\phi\) (que corresponde al desfase tensión-corriente de la carga):
\[
    i_f(t) = I_{max\_f} \sin(\omega t) \quad \text{y} \quad i_m(t) = I_{max\_m} \sin(\omega t - \phi)
\]

La cupla instantánea oscila al doble de la frecuencia de la red (100 Hz o 120 Hz). Sin embargo, el sistema móvil tiene \textbf{inercia mecánica} y no puede seguir esas oscilaciones tan rápidas. Por tanto, la aguja toma la posición correspondiente al \textbf{valor medio} de la cupla:

\[
    (C_m)_{\text{med}} = \frac{1}{T} \int_0^T i_f \cdot i_m \cdot \frac{dM}{d\alpha} \, dt
\]

Resolviendo la integral del producto de senos, obtenemos que el valor medio depende de los valores eficaces (\(I_{rms}\)) y el coseno del ángulo de desfase:

\[
    (C_m)_{\text{med}} = I_F \cdot I_M \cdot \cos(\phi) \cdot \frac{dM}{d\alpha}
\]

Igualando con el resorte (\(k \cdot \alpha\)), obtenemos la \textbf{ley de deflexión para CA}:

\begin{equation}
    \alpha = \frac{I_F \cdot I_M \cdot \cos(\phi)}{k} \cdot \frac{dM}{d\alpha}
\end{equation}

\subsection{Conclusión para el Vatímetro}
Si aplicamos lo anterior al uso como vatímetro:
\begin{itemize}
    \item La corriente fija es la de línea: \(I_F = I\).
    \item La corriente móvil es proporcional a la tensión: \(I_M = \frac{V}{R_V}\).
\end{itemize}

Sustituyendo en la ecuación de CA:
\[
    \alpha = \frac{I \cdot (V/R_V) \cdot \cos(\phi)}{k} \cdot \frac{dM}{d\alpha} = \left( \frac{1}{k \cdot R_V} \frac{dM}{d\alpha} \right) \cdot (V \cdot I \cdot \cos \phi)
\]

\[
    \alpha = K \cdot P_{\text{activa}}
\]

La deflexión es \textbf{linealmente proporcional a la Potencia Activa}.

\section{Aplicación como Amperímetro y Voltímetro}

La característica principal de usar el instrumento electrodinámico para medir corriente o tensión es que permite obtener el \textbf{Valor Eficaz Verdadero} (True RMS), tanto para CC como para CA, incluso si la forma de onda no es senoidal pura.

Esto ocurre porque la deflexión depende del cuadrado de la magnitud medida.

\subsection{Como Amperímetro}
\textbf{Conexión:}
Las dos bobinas (fija y móvil) se conectan en \textbf{serie} entre sí y en serie con la carga.

\textbf{¿Por qué?}
Necesitamos que la misma corriente \(I\) que genera el campo magnético (en la fija) sea la que atraviesa la bobina móvil.
\[
    \alpha \propto I_{\text{fija}} \cdot I_{\text{móvil}} = I \cdot I = I^2
\]
Al depender de \(I^2\), la aguja siempre deflecta en el mismo sentido (positivo), sin importar si la corriente es positiva o negativa.

\textbf{Características:}
\begin{itemize}
    \item Para corrientes pequeñas (miliamperios), pasan directamente por las bobinas.
    \item Para corrientes grandes, se suelen poner las bobinas en paralelo entre sí (con shunts), pero el principio sigue siendo que ambas reciben una proporción de la corriente total.
    \item La escala es \textbf{cuadrática} (comprimida al inicio, expandida al final).
\end{itemize}

\subsection{Como Voltímetro}
\textbf{Conexión:}
Las dos bobinas (fija y móvil) se conectan en \textbf{serie} entre sí, y a su vez en serie con una gran resistencia limitadora (R\textsubscript{s}). Todo el conjunto se conecta en paralelo a la fuente o carga a medir.

\textbf{¿Por qué?}
La resistencia transforma la tensión \(V\) en una corriente pequeña \(i\) proporcional (\(i = V/R\)). Al estar las bobinas en serie, esa misma corriente recorre ambas.
\[
    \alpha \propto i \cdot i \propto \left(\frac{V}{R}\right) \cdot \left(\frac{V}{R}\right) \propto V^2
\]
De nuevo, al medir \(V^2\), obtenemos el valor eficaz de la tensión.

\textbf{Características:}
\begin{itemize}
    \item Consume bastante potencia propia comparado con otros voltímetros (debido a la corriente necesaria para crear el campo magnético).
    \item Al igual que el amperímetro, tiene una escala cuadrática.
\end{itemize}

\subsection{Vatímetro}

Un vatímetro es un instrumento que presenta cuatro bornes. Dos de ellos corresponden a la bobina voltimétrica y los otros dos a la amperométrica. Generalmente se encuentran perfectamente identificados. En algunos casos el tamaño de los bornes de conexión de la bobina amperométrica es mayor habida cuenta que estos deben manejar un importante valor de corriente.

Los vatímetros son instrumentos caros y su costo se incrementa en función de la clase, siendo común encontrarlos en clases 0,1; 0,2; 0,5; 1 y 2.

En general, se puede apreciar que el principio de funcionamiento de estos instrumentos es muy similar a los de ipbm, salvo que el campo magnético que producía el imán permanente, es ahora producido por otro bobinado, lo cual les da la posibilidad de funcionar tanto en corriente continua como en corriente alterna, por cuanto las polaridades se invierten de forma simultánea en ambos bobinados.

\subsection{Vatímetro compensado}

\subsection{Error de fase}

\subsection{Vatímetro compensado en fase}

\subsection{Constante del vatímetro}

\subsection{Precauciones en la conexión}

\subsection{Polaridad}

\subsection{Varímetro}

\subsection{Fasímetro}

\subsection{Cofímetro trifásico}

\subsection{Frecuencímetro}

\subsection{Errores sistemáticos}

\subsection{Instrumentos Astáticos}

\subsection{Medición del verdadero valor eficaz}
